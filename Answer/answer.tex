\documentclass[UTF8]{ctexart}
\usepackage{verbatim,amsthm,amsfonts,mathdots}
\usepackage{xeCJK,geometry,float,graphicx}
\usepackage{amsmath,amssymb,zhnumber,booktabs,setspace,tasks}
\usepackage{cases}
\usepackage{cite}
\usepackage{fancyhdr}
\usepackage{multirow}
\geometry{a4paper}
\pagestyle{fancy}
\fancyhf{}


\title{妮可代数结构答案}
\author{En土土}
\date{\today}
\pagenumbering{arabic}

\begin{document}

\fancyhead[L]{En土土}
\fancyhead[C]{代数结构答案}
\fancyhead[R]{妮可}
\fancyfoot[C]{\thepage}

\maketitle
\tableofcontents
\newpage

\section{集合}
\subsection{}   %1
\begin{enumerate}
    \item [(1)]不相等.
    \item [(2)]相等.
    \item [(3)]相等.
\end{enumerate}

\subsection{}   %2
\begin{proof}
    \[
        \begin{cases}
            \ A \subseteq B \Rightarrow \ \forall\ x\in A,\ x\in B. \\
            \\
            \ B \subset C \Rightarrow
            \ 
            \begin{cases}
                \ \forall\ x\in B,\ x\in C \\
                \ \exists\ x\in C,\ x\notin B
            \end{cases}
        \end{cases}
        \Rightarrow\ 
        \begin{cases}
            \ \forall\ x\in A,\ x\in C\\
            \ \exists\ x\in C,\ x\notin A
        \end{cases}
        \ \Rightarrow\ 
        A \subset C.
    \]
\end{proof}

\subsection{}   %3
\begin{enumerate}
    \item [(1)]不成立.
    \item [(2)]不成立.
    \item [(3)]不成立.
    \item [(4)]成立.
    \item [(5)]成立.
    \item [(6)]不成立.
\end{enumerate}

\subsection{}   %4
\begin{enumerate}
    \item [(1)]不成立.
    \item [(2)]成立.
    \item [(3)]成立.
\end{enumerate}

\subsection{}   %5
\begin{proof}
    \begin{enumerate}
        \item []
        \item [(1)]
        \[
            A\cap \left( \overline{A} \cup B\right)
            =
            \left( A \cap \overline{A}\right) \cup \left( A \cap B \right)
            =
            \varnothing  \cup \left(A \cap B\right)
            = 
            A \cap B.
        \]
        \item [(2)]
        \[
            A\cup \left(A\cap B\right)
            =
            \left(A \cup A\right) \cap \left(A\cup B\right)
            =
            A\cap \left(A\cup B\right).
        \]
        \[
           \begin{cases}
            \ A\subseteq A\cup \left(A\cap B\right)\\
            \ A\supseteq A\cap \left(A\cup B\right)
           \end{cases}
           \ \Rightarrow\ 
           A\cup \left(A\cap B\right)
           = A.
        \]
        \item [(3)]
        \begin{enumerate}
            \item [(a)]
                \begin{align*}
                    \forall\ x \in \overline{\bigcap \limits_{i} A_i} \ \Rightarrow\ & x \notin  \bigcap \limits_{i} A_i & \forall\ x \in \bigcup\limits_{i}\overline{A_i} \ \Rightarrow\  & \exists 1 \leq k \leq n,x \in \overline{A_k}\\
                    \Rightarrow\ & \exists 1 \leq k \leq n,x \notin A_k & \Rightarrow\ & \exists 1 \leq k \leq n,x \notin A_k\\
                    \Rightarrow\ & \exists 1 \leq k \leq n,x \in \overline{A_k} & \Rightarrow\ & x \notin \bigcap \limits_{i}A_i\\
                    \Rightarrow\ & x \in \bigcup\limits_{i}\overline{A_i} & \Rightarrow\ & x \in \overline{\bigcap \limits_{i} A_i}\\
                    \Rightarrow\ & \overline{\bigcap \limits_{i} A_i} \subseteq \bigcup\limits_{i}\overline{A_i} & \Rightarrow\ & \bigcup\limits_{i}\overline{A_i} \subseteq \overline{\bigcap \limits_{i} A_i}
                \end{align*}
                即证$\overline{\bigcap \limits_{i} A_i}=\bigcup\limits_{i}\overline{A_i}$.
            \item [(b)]
            \begin{align*}
                \forall\ x \in \overline{\bigcup \limits_{i} A_i} \ \Rightarrow\ & x \notin  \bigcup \limits_{i} A_i & \forall\ x \in \bigcap\limits_{i}\overline{A_i} \ \Rightarrow\ & \forall 1 \leq k \leq n,x \in \overline{A_k}\\
                \Rightarrow\ & \forall 1 \leq k \leq n,x \notin A_k & \Rightarrow\ & \forall 1 \leq k \leq n,x \notin A_k\\
                \Rightarrow\ & \forall 1 \leq k \leq n,x \in \overline{A_k} & \Rightarrow &\ x \notin  \bigcup \limits_{i} A_i\\
                \Rightarrow\ & x \in \bigcap \limits_{i}\overline{A_i} & \Rightarrow\ & x \in \overline{\bigcup \limits_{i} A_i}\\
                \Rightarrow\ & \overline{\bigcup \limits_{i} A_i} \subseteq \bigcap\limits_{i}\overline{A_i} & \Rightarrow\ & \bigcap\limits_{i}\overline{A_i} \subseteq \overline{\bigcup \limits_{i} A_i}
            \end{align*}
            即证$\overline{\bigcup \limits_{i} A_i}=\bigcap\limits_{i}\overline{A_i}$.
        \end{enumerate}
    \end{enumerate}
\end{proof}

\subsection{}   %6
\begin{proof}
    \begin{enumerate}
        \item []
        \item [(1)]
        $B \subseteq C \Rightarrow \forall x \in B,x \in C$.
        \[
            \forall x \in (A \cap B),x \in A \mbox{且} \ x \in B
            \ \Rightarrow\ x \in A \mbox{且} \ x \in C
            \ \Rightarrow\ x \in (A \cap C)
        \]
        \item [(2)]
        \begin{align*}
            A\subseteq C,\ B\subseteq\ C 
            \ \Leftrightarrow\ &
            A\cup C=C,\ B\cup C=C\\
            \ \Leftrightarrow\ &
            \left(A\cup B\right)\cup C
            =A\cup \left(B\cup C\right)
            =A\cup C
            =C\\
            \ \Leftrightarrow\ &
            \left(A\cup B\right) \subseteq C.
        \end{align*}
        \item [(3)]
        若$|A \cup B| > |A| + |B|$,则$\exists x \in (A\cup B)$,且$x \notin A,x \notin B$,矛盾.

        $|A \cup B| = |A| + |B| - |A \cap B|$,$|A \cup B| = |A| + |B|$当且仅当$A \cap B =\phi $时.
    \end{enumerate}
\end{proof}

\subsection{}   %7
\begin{enumerate}
    \item [(1)]设所求集合为$E$.
    \begin{enumerate}
        \item [1.](基础语句)令$D=\left\{0,1,2,3,4,5,6,7,8,9\right\}$,若$x\in\ D$,则$x\in\ E$.
        \item [2.](归纳语句)若$x,\ y\in\ E$,则$x$与$y$的连接$\overline{xy}\in\ E$.
        \item [3.](终结语句)$x\in \ E$,当且仅当$x$是由有限次$1,2$得到的.
    \end{enumerate}

    \item [(2)]设所求集合为$E$.
    \begin{enumerate}
        \item [1.](基础语句)令$D=\left\{0,1,2,3,4,5,6,7,8,9\right\}$,若$x\in\ D$,则$x.\in\ E,\ .x\in\ E$.
        \item [2.](归纳语句)若$x=a.b,\ y=c.d\in\ E$,则$\overline{ac} . \overline{bd}\in\ E$.
        \item [3.](终结语句)$x\in \ E$,当且仅当$x$是由有限次$1,2$得到的.
    \end{enumerate}

    \item [(3)]设所求集合为$E$.
    \begin{enumerate}
        \item [1.](基础语句)$0\in\ E$.
        \item [2.](归纳语句)若$x\in\ E$,则$x+10\in \ E$.
        \item [3.](终结语句)$x\in \ E$,当且仅当$x$是由有限次$1,2$得到的.
    \end{enumerate}
\end{enumerate}

\newpage

\section{数论初步}

\newpage

\section{映射}

\newpage

\section{二元关系}

\newpage

\section{群论初步}

\newpage

\section{商群}

\newpage

\section{环和域}

\newpage

\section{格和布尔代数}

\end{document}