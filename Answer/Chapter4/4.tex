\documentclass[UTF8]{ctexart}
\usepackage{verbatim,amsthm,amsfonts,mathdots}
\usepackage{xeCJK,geometry,float,graphicx}
\usepackage{amsmath,amssymb,zhnumber,booktabs,setspace,tasks,booktabs}
\usepackage{tabularray}
\usepackage{cases}
\usepackage{cite}
\usepackage{fancyhdr}
\usepackage{multirow}
\usepackage{slashed}
\geometry{a4paper}
\pagestyle{fancy}
\fancyhf{}
\setlength{\tabcolsep}{8pt} % Default value: 6pt

\pagenumbering{arabic}

\begin{document}

\fancyhead[L]{En土土}
\fancyhead[C]{代数结构答案}
\fancyhead[R]{妮可}
\fancyfoot[C]{\thepage}

\section{二元关系}
\subsection{}   %1
\begin{enumerate}
    \item [(1)]$R_1$有对称性.
    \item [(2)]$R_2$有对称性.
    \item [(3)]$R_3$有传递性,反自反性.
    \item [(4)]$R_4$有自反性,传递性,反对称性.
    \item [(5)]$R_5$有自反性,对称性,传递性.
\end{enumerate}

\subsection{}   %2
\begin{enumerate}
    \item [(1)]$a R_1 b$,当且仅当$ab \geq 0$,有
    \[
        \begin{cases}
            &\forall\ a\in \mathbb{Z},\ a R_1 a;\\
            &a R_1 b \ \leftrightarrow\ ab\geq 0 \ \leftrightarrow\ b R_1 a;\\
            &(-1) R_1 0\ ,\ 0 R_1 1\ ,\ (-1)\slashed{R}_1 1 .
        \end{cases}
    \]
    \item [(2)]$a R_2 b$,当且仅当$a \geq b$,有
    \[
        \begin{cases}
            &\forall\ a\in \mathbb{Z},\ a R_2 a;\\
            &a R_2 b,\ b R_2 c \ \Rightarrow\ a\geq b \geq c \ \Rightarrow\ a R_2 c;\\
            &5 R_2 1\ , \ 1\slashed{R}_2 5 .
        \end{cases}
    \]
    \item [(3)]$a R_3 b$,当且仅当$ab > 0$,有
    \[
        \begin{cases}
            &a R_3 b \ \Rightarrow\ ab>0 \ \Rightarrow\ b R_3 a;\\
            &a R_3 b,\ b R_3 c \ \Rightarrow\ ab>0,\ bc>0,\ ac=abcd/b^2>0 \ \Rightarrow\ a R_3 c;\\
            &0\in\mathbb{Z},\ 0 \slashed{R}_3 0.
        \end{cases}
    \]
\end{enumerate}

\subsection{}   %3
\begin{enumerate}
    \item [(1)]$R_1 \circ R_2 = \{(c,d) \}$
    \item [(2)]$R_2 \circ R_1 = \{(a,d), (a,c)\}$
    \item [(3)]$R_1^2 = \{(a,a), (a,b), (a,d)\}$ 
    \item [(4)]$R_2^3 = \{(b,c), (b,d), (c,d)\}$
\end{enumerate}

\subsection{}   %4
\begin{proof}
    \begin{enumerate}
        \item []
        \item []$\forall\ (a,c)\in R_1 \circ (R_2 \cap R_3)$,则
        \[
            \exists\ (a,b)\in (R_2 \cap R_3),\ (b,c)\in R_1 .    
        \]
        故
        \[
            \begin{cases}
                (a,b) & \in R_2 \\
                (a,b) & \in R_3
            \end{cases}    
            \ \Rightarrow\ 
            \begin{cases}
                (a,c) & \in R_1 \circ R_2\\
                (a,c) & \in R_1 \circ R_3
            \end{cases}
            \ \Rightarrow\ 
            (a,c)\in \left((R_1 \circ R_2) \cap (R_1 \circ R_3) \right)
        \]
        即证$R_1 \circ (R_2 \cap R_3) \subseteq (R_1 \circ R_2) \cap (R_1 \circ R_3)$.
    \end{enumerate}
\end{proof}

\subsection{}   %5
\begin{proof}
    \begin{enumerate}
        \item []
        \item [(1)]$\forall\ x\in A,\ (x,x)\in I_A$,故$(x,x)\in R'$,即证$R'$在A上自反.
        \item [(2)]$R \subseteq I_A \cup R\ \Rightarrow\ R\subseteq R'$.
        \item [(3)]若有自反关系$R''$满足$R \subseteq R''$,由自反性可得$I_A \subseteq R''$,故
        \[
            R'=I_A \cup R \subseteq R''    
        \]
        即证$R'$为$R$的自反闭包.
    \end{enumerate}
\end{proof}

\subsection{}   %6
\begin{enumerate}
    \item [(1)]
    \begin{proof}
        \begin{enumerate}
            \item [(a)]自反性:
            \[
                \forall\ (a,b)\in \mathbb{N}\times \mathbb{N},\ 
                a+b = b+a\ \Rightarrow\ (a,b)\sim (a,b).    
            \]
            \item [(b)]对称性:
            \[
                \forall\ (a,b)\sim (c,d),\ \mbox{有}
                a+d = b+c,\ c+b = a+d
                \ \Rightarrow\ 
                (c,d)\sim (a,b).    
            \]
            \item [(c)]传递性:
            \[
                \forall\ (a,b)\sim (c,d),\ (c,d)\sim (e,f),\ \mbox{有}
                a+d = b+c,\ c+f = d+e
            \]
            故
            \[
                a+f = (a+d)+(c+f)-d-c = (b+c)+(d+e)-d-c = b+e
                \ \Rightarrow\ 
                (a,b)\sim (e,f).
            \]
        \end{enumerate}
    \end{proof}
    \item [(2)]
\end{enumerate}

\subsection{}   %7



\subsection{}   %8



\subsection{}   %9



\subsection{}   %10



\subsection{}   %11



\subsection{}   %12



\subsection{}   %13



\subsection{}   %14



\subsection{}   %15



\subsection{}   %16



\subsection{}   %17



\subsection{}   %18



\subsection{}   %19



\subsection{}   %20



\end{document}