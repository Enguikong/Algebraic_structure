\documentclass[UTF8]{ctexart}
\usepackage{verbatim,amsthm,amsfonts,mathdots}
\usepackage{xeCJK,geometry,float,graphicx}
\usepackage{amsmath,amssymb,zhnumber,booktabs,setspace,tasks,booktabs}
\usepackage{tabularray}
\usepackage{cases}
\usepackage{cite}
\usepackage{fancyhdr}
\usepackage{multirow}
\geometry{a4paper}
\pagestyle{fancy}
\fancyhf{}
\setlength{\tabcolsep}{8pt} % Default value: 6pt

\pagenumbering{arabic}

\begin{document}

\fancyhead[L]{En土土}
\fancyhead[C]{代数结构答案}
\fancyhead[R]{妮可}
\fancyfoot[C]{\thepage}

\section{映射}
\subsection{}   %1
\begin{enumerate}
    \item [(1)]不能构成映射,如$x_1=0$,则$x_2=0,1,\ldots,9$均满足.
    \item [(2)]能构成映射,$\forall\ y_1\in \mathbb{R},\ \exists !\ y_2 = y_1^2$.
    \item [(3)]不能构成映射,如$y_1 = 1$,则$y_2 = \pm 1$均满足.
\end{enumerate}

\subsection{}   %2
\begin{enumerate}
    \item [(1)]
    \[
        R_f = \{ 2 , -2 , 0 \}    
    \]
    \item [(2)]
    \[
        \begin{cases}
            |A| = 9\\
            |R_f| = 3
        \end{cases}    
        \ \Rightarrow\ 
        n = 3^9.
    \]
\end{enumerate}

\subsection{}   %3
\begin{enumerate}
    \item [(1)]满射
    \[
        \forall\ y\in \mathbb{Z}^{+},\ 
        \exists\ x=\pm(y-1)\in \mathbb{Z},\ 
        f(x)=y.    
    \]
    \item [(2)]既不是单射,又不是满射.值域为$\{0,1,2\}\subseteq \mathbb{Z}\cup\{0\}$
    \[
        \forall\ y\in \mathbb{Z}\cup\{0\}
        \begin{cases}
            y\in \{0,1,2\},\ \exists\ x=3k+y(k\in \mathbb{Z}),\ f(x)=y.\\
            y\notin \{0,1,2\},\ \forall\ x\in \mathbb{Z},\ f(x)\neq y.
        \end{cases}
    \]
    \item [(3)]$f,g$均是双射.
    \[
        \begin{cases}
            \forall\ y\in \mathbb{Z},\ \exists !\ x=y-1\in \mathbb{Z},f(x)=y.\\
            \forall\ y\in \mathbb{Z},\ \exists !\ x=y+1\in \mathbb{Z},g(x)=y. 
        \end{cases}    
    \]
    \item [(4)]满射.
    \[
        \forall\ y\in \{0,1\},\ 
        \exists\ x=2k+y+1(k\in\mathbb{Z}),\ f(x)=y.    
    \]
    \item [(5)]既不是单射,又不是满射.
    \[
        \begin{cases}
            \forall\ y\in \mathbb{Z},\ y<-16,\ \forall\ x\in\mathbb{Z},\ f(x)>y.\\
            \exists\ y=-15,\ f(0)=f(-2)=y.
        \end{cases}
    \]
\end{enumerate}

\subsection{}   %4
\begin{proof}
    \[
        \begin{cases}
            f:A\times B \rightarrow B\times A,\ (a,b) \mapsto (b,a) \\
            g:B\times A \rightarrow A\times B,\ (b,a) \mapsto (a,b) 
        \end{cases}    
    \]
    则
    \[
        f\circ g = {I}_{A\times B},\quad
        g\circ f = {I}_{B\times A},\quad
        g=f^{-1}. 
    \]
    即$f$为双射,即证$|A\times B|=|B\times A|$.
\end{proof}

\subsection{}   %5
\begin{proof}
    \begin{enumerate}
        \item []
        \item [(1)]
        \[
            \forall\ f(x)=\sum\limits_{i=0}^{n} a_i\cdot x^i,\ 
            \exists !\ g(x)=\sum\limits_{i=1}^{n} i\cdot a_i \cdot x^{i-1},\ 
            \frac{d}{dx}f(x)=g(x).\ (a_i \in\mathbb{R})
        \]
        值域为$R[x]$,是满射不是双射.
        \[
            \forall\ g(x)=\sum\limits_{i=0}^{n} a_i\cdot x^i,\ 
            \exists\ f(x)=a+\sum\limits_{i=0}^{n} \frac{a_i}{i+1} \cdot x^{i+1},\ 
            \frac{d}{dx}f(x)=g(x).\ (a,a_i \in\mathbb{R})
        \]
        \item [(2)]
        \[
            \forall\ f(x)=\sum\limits_{i=0}^{n} a_i\cdot x^i,\ 
            \exists !\ g(x)=\sum\limits_{i=0}^{n} \frac{a_i}{i+1} \cdot x^{i+1},\ 
            I(f(x))=g(x).\ (a_i \in\mathbb{R})
        \]
        值域为常数项为0的实系数多项式,既不是满射,也不是双射.
        \[
            \forall\ g(x)=a\ (0\neq a\in\mathbb{R}),\ 
            \mbox{若}\exists\ f(x)\in R[x],\mbox{满足} I(f(x))=g(x)=a,
            I(f(x))=g(x).
        \]
        则
        \[
            a = g(x) = \int_{0}^{x} f(t) \,dt  
            \ \xrightarrow{x=0}\ 
            g(0) = \int_{0}^{0} f(t) \,dt = 0 \neq a.  
        \]
        矛盾.
    \end{enumerate}
\end{proof}

\subsection{}   %6
\begin{proof}
    \begin{enumerate}
        \item []$\forall\ (b_{i1}, b_{i2}, \ldots, b_{in}) \in S(B)$,
        \[
            \ \exists !\ f\in F,\ f: A \rightarrow B,\ a_j \mapsto b_{ij}. 
            \mbox{满足} g(f) = (b_{i1}, b_{i2}, \ldots, b_{in}) .
        \]
        即证$g$即是单射,又是满射,即$g$是从$F$到$S(B)$的双射.
        \[
            |F| = |S(B)| = {|B|}^{n} = m^{n}.
        \]
    \end{enumerate}
\end{proof}

\subsection{}   %7
\begin{proof}
    \begin{enumerate}
        \item []
        \item [(1)]$f(A\cup B) = f(A) \cup f(B)$
        \begin{enumerate}
            \item [(a)]$\forall\ y\in f(A\cup B),\ \exists\ x\in A\cup B,\ f(x)=y$.
            \[
                \begin{cases}
                    x\in A , y=f(x)\in f(A)\\
                    x\in B , y=f(x)\in f(B)
                \end{cases}  
                \ \Rightarrow \
                y\in f(A)\cup f(B)
                \ \Rightarrow \
                f(A\cup B) \subseteq f(A)\cup f(B).
            \]
            \item [(b)]$\forall\ y\in f(A)\cup f(B),\ y\in f(A)\mbox{或}y\in f(B)$.
            \[
                \begin{cases}
                    y\in f(A) , \exists\ x\in A, f(x) = y \\
                    y\in f(B) , \exists\ x\in B, f(x) = y
                \end{cases}  
                \ \Rightarrow \
                y\in f(A\cup B)
                \ \Rightarrow \
                f(A)\cup f(B) \subseteq f(A\cup B).
            \]
        \end{enumerate}
        综上,即证
        \[
            f(A\cup B) = f(A) \cup f(B).
        \]
        \item [(2)]$f(A\cap B) \subseteq f(A) \cap f(B)$
        \[
            \forall\ y\in f(A\cap B),\ \exists\ x\in A\cap B,\ f(x)=y.    
        \]
        由$x\in A\cap B$有$x\in A,x\in B$,即
        \[
            \begin{cases}
                x\in A,\ y=f(x)\in f(A)\\
                x\in B,\ y=f(x)\in f(B)
            \end{cases}    
            \ \Rightarrow \ 
            y\in f(A) \cap f(B)
            \ \Rightarrow \ 
            f(A\cap B) \subseteq f(A)\cap f(B).
        \]
        \item [(3)]
        \[
            S = \mathbb{Z}, T=\{1\}, 
            A=\{2k+1 | k\in \mathbb{Z}\}, B=\{2k | k\in \mathbb{Z}\}
        \]
        取$f:S\rightarrow T, n \mapsto 1$,则
        \[
            \begin{cases}
                f(A\cap B) = f(\phi) = \phi.\\
                f(A)\cap f(B) = \{1\}
            \end{cases}  
            \quad \Rightarrow\quad 
            f(A\cap B)\neq f(A) \cap f(B).  
        \]
    \end{enumerate}
\end{proof}

\subsection{}   %8
\begin{enumerate}
    \item [(1)]$f$为单射:$f(\widetilde{A})\subseteq \widetilde{f(A)}$.
    \[
        \forall\ y\in f(\widetilde{A}) \subseteq S,\ 
        \exists !\ x\in S\mbox{且} x\in \widetilde{A},\ f(x)=y.  
    \]
    有
    \[
        \forall\ x\in A,\ f(x)\neq y,\ y\notin f(A) 
        \ \Rightarrow\ 
        y\in \widetilde{f(A)}. 
    \]
    即
    \[
        \forall\ y\in f(\widetilde{A}) \subseteq S,\ 
        y\in \widetilde{f(A)}
        \ \Rightarrow\ 
        f(\widetilde{A})\subseteq \widetilde{f(A)}.
    \]
    \item [(2)]$f$为满射:$\widetilde{f(A)}\subseteq f(\widetilde{A})$.
    \[
        \forall\ y\in \widetilde{f(A)},\ 
        \exists\ x\in S,f(x)=y.    
    \]
    若$x\in A,\ y=f(x)\in f(A)$,矛盾.故$x\in \widetilde{A}$.
    \[
        y=f(x)\in f(\widetilde{A})
        \ \Rightarrow\ 
        \widetilde{f(A)}\subseteq f(\widetilde{A}).
    \]
\end{enumerate}

\subsection{}   %9
\begin{enumerate}
    \item [(1)]
    \[
        f\circ g
        =
        3(3x+1)
        =
        9x+3.
    \]
    \item [(2)]
    \[
        f\circ g
        =
        3    
    \]
    \item [(3)]
    \[
        g\circ f
        =
        3(3x)+1
        =
        9x+1.
    \]
    \item [(4)]
    \[
        g\circ h
        =
        3(3x+2)+1
        =
        9x+7.
    \]
    \item [(5)]
    \[
        f\circ g \circ h
        =
        3(3(3x+2)+1)
        =
        27x+21.   
    \]
\end{enumerate}

\subsection{}   %10
\begin{proof}
    
    \begin{enumerate}
        \item []
        \item []先证$g\circ f$是从$A$到$C$的映射.
        \[
            \forall\ x\in A,\ \exists !\ y=f(x)\in B,\ \exists !\ z=g(y)\in C.
        \]
        \item []假设$g\circ f$不是从$A$到$C$的单射,则等价于
        \[
            \exists\ c\in C,\ \exists\ a_1,a_2\in A,\ a_1\neq a_2,\
            g\circ f(a_1)=g\circ f(a_2)=c.
        \]
        即
        \[
            g(f(a_1))=g(f(a_2))=c\in C
            \ \xrightarrow{g为单射}\ 
            \exists !\ b\in B,\ g(b)=c,\ 
            f(a_1)=f(a_2)=b\in B.    
        \]
        又
        \[
            f\mbox{为单射}
            \ \Rightarrow\ 
            \exists\ a\in A,\ f(a)=b.    
        \]
        即$a_1 = a_2 = a$,矛盾,即证$g\circ f$是从$A$到$C$的单射.
    \end{enumerate}
\end{proof}

\subsection{}   %11
\begin{enumerate}
    \item []会发生矛盾.
    \item [(1)]先证$g$为满射
    \[
        \mbox{若}\exists\ y\in S,\ \forall\ x\in S,\ g(x)\neq y,\    
        \mbox{则}f(y)\in S,\ \mbox{且}f(y)\notin f\circ g (S),\ \mbox{矛盾}.
    \]

    \item [(2)]再证$g$为单射
    \[
        \mbox{若}\exists\ a,b\in S,\ a\neq b,\ \mbox{且}g(a)=g(b),\ 
        \mbox{则}f(g(a))=f(g(b)).
    \]
    又$f\circ\ g (a)=a,\ f\circ g(b)=b,\ a\neq b$,矛盾.
    \item []综上,即证$g$为双射.故
    \[
        \forall\ x\in\ S,\ f(g(x))=f(f^{-1}(x))=x,\ 
        \ \Rightarrow\ 
        \forall\ x\in\ S,\ g(x)=f^{-1}(x),\ g\circ f=I_S.    
    \]
    故矛盾.
\end{enumerate}

\subsection{}   %12
\begin{enumerate}
    \item [(1)]
    \[
        \tau \sigma 
        = 
        \begin{pmatrix}
            1 & 2 & 3 & 4 & 5 & 6\\
            1 & 2 & 3 & 6 & 5 & 4\\
        \end{pmatrix}  
        = 
        (1)(2)(3)(4\ 6)(5).
    \]
    \item [(2)]
    \[
        \tau ^2 \sigma 
        = 
        \begin{pmatrix}
            1 & 2 & 3 & 4 & 5 & 6\\
            2 & 4 & 1 & 5 & 6 & 3
        \end{pmatrix}  
        =
        (124563).
    \]
    \item [(3)]
    \[
        \sigma ^2 \tau  
        = 
        \begin{pmatrix}
            1 & 2 & 3 & 4 & 5 & 6\\
            3 & 1 & 4 & 5 & 6 & 2
        \end{pmatrix}
        \begin{pmatrix}
            1 & 2 & 3 & 4 & 5 & 6\\
            1 & 5 & 3 & 4 & 2 & 6
        \end{pmatrix}
        =
        \begin{pmatrix}
            1 & 2 & 3 & 4 & 5 & 6\\
            3 & 6 & 4 & 5 & 1 & 2
        \end{pmatrix}.
    \]
    \item [(4)]
    \[
        \sigma ^{-1} \tau \sigma 
        = 
        \begin{pmatrix}
            1 & 2 & 3 & 4 & 5 & 6\\
            2 & 6 & 1 & 3 & 4 & 5\\
        \end{pmatrix} 
        \begin{pmatrix}
            1 & 2 & 3 & 4 & 5 & 6\\
            1 & 2 & 3 & 6 & 5 & 4\\
        \end{pmatrix}  
        =
        \begin{pmatrix}
            1 & 2 & 3 & 4 & 5 & 6\\
            2 & 6 & 1 & 5 & 4 & 3
        \end{pmatrix}.
    \]
\end{enumerate}

\subsection{}   %13
\begin{enumerate}
    \item [(1)]
    \[
        (257)(78)(145)
        =
        \begin{pmatrix}
            1 & 2 & 4 & 5 & 7 & 8\\
            4 & 5 & 7 & 1 & 8 & 2
        \end{pmatrix}  
        =
        (147825).
    \]
    \item [(2)]
    \[
        (72815)(21)(476)(12)
        =
        \begin{pmatrix}
            1 & 2 & 4 & 5 & 6 & 7 & 8\\
            5 & 8 & 2 & 7 & 4 & 6 & 1
        \end{pmatrix}  
        =
        (1576428).
    \]
\end{enumerate}

\subsection{}   %14
\begin{enumerate}
    \item [(1)]
    \[
        \begin{pmatrix}
            1 & 2 & 3 & 4 & 5 & 6 & 7 & 8\\
            8 & 2 & 6 & 3 & 7 & 4 & 5 & 1
        \end{pmatrix}
        =
        (18)(2)(364)(57). 
    \]
    \item [(2)]
    \[
        \begin{pmatrix}
            1 & 2 & 3 & 4 & 5 & 6 & 7 & 8\\
            3 & 6 & 4 & 1 & 8 & 2 & 5 & 7
        \end{pmatrix}  
        =(134)(26)(587).  
    \]
    \item [(3)]
    \[
        \begin{pmatrix}
            1 & 2 & 3 & 4 & 5 & 6 & 7 & 8\\
            3 & 1 & 4 & 7 & 2 & 5 & 8 & 6
        \end{pmatrix}
        =(13478652). 
    \]
\end{enumerate}

\subsection{}   %15
\begin{enumerate}
    \item [(1)]
    \[
        (47)(261)(567)(1234)
        =
        \begin{pmatrix}
            1 & 2 & 3 & 4 & 5 & 6 & 7\\
            6 & 3 & 7 & 2 & 1 & 4 & 5
        \end{pmatrix}   
        =
        (1642375). 
    \]
    故阶为7.
    \item [(2)]
    \[
        (163) (1357) (67) (12345)   
        =
        \begin{pmatrix}
            1 & 2 & 3 & 4 & 5 & 6 & 7\\
            2 & 5 & 4 & 7 & 1 & 6 & 3
        \end{pmatrix}
        =
        (125)(347)(6). 
    \]
    $[3,3,1]=3$,故阶为3.
\end{enumerate}

\subsection{}   %16
\begin{proof}
    \begin{enumerate}
        \item []
        \item [(1)]由定理$3.9$,任何$n$元置换都可以表示成若干不相交的轮换之乘积.
        \item [(2)]任意轮换可以表示成对换之积.
        \item [(3)]任意对换$\sigma=(ij)\ (j>i)$,有
        \[
            (ij)= (i (i+1)) ((i+1)(i+2))\ldots ((j-1)j).
        \]
        \item []综上,即证任意置换都可以表示成对换$(12),\ldots,((n-1)n)$的乘积.
    \end{enumerate}
\end{proof}

\subsection{}   %17
\begin{proof}
    \begin{enumerate}
        \item []
        \item [(1)]
        \begin{align*}
            x_1 x_2 x_3 + x_1 \overline{x}_2 x_3 + x_1  + x_1 \overline{x}_2 \overline{x}_3
            & =
            x_1 \left(x_2 x_3 + \overline{x}_2 x_3\right)+ x_1 \left(x_2 \overline{x}_3 + \overline{x}_2 \overline{x}_3 \right)\\
            & =
            x_1 x_3 \left(x_2 + \overline{x}_2 \right)+ x_1 \overline{x}_3 \left(x_2 + \overline{x}_2 \right)\\
            & =
            x_1 x_3 + x_1 \overline{x}_3\\
            & =
            x_1 \left(x_3 + \overline{x}_3 \right)\\
            & =
            x_1 .
        \end{align*}
        \item [(2)]
        \begin{align*}
            x_1 x_2 + x_2 x_3 + \overline{x}_1 x_3
            & =
            x_1 x_2 + (x_1 + \overline{x}_1)x_2 x_3 + \overline{x}_1 x_3\\
            & =
            x_1 x_2 + x_1 x_2 x_3 + \overline{x}_1 x_2 x_3 + \overline{x}_1 x_3\\
            & =
            x_1 x_2 (1 + x_3) + \overline{x}_1 x_3 (1 + x_2)\\
            & =
            x_1 x_2 + \overline{x}_1 x_3 .
        \end{align*}
    \end{enumerate}    
\end{proof}


\subsection{}   %18
\begin{proof}
    \begin{enumerate}
        \item [(1)]
        \begin{align*}
            f\cdot g + \overline{f}
            & =
            f\cdot (f+g) + \overline{f}\\
            & =
            f\cdot f + f\cdot g + \overline{f}\\
            & =
            f + \overline{f} + f\cdot g\\
            & =
            1 + f\cdot g\\
            & =
            1
        \end{align*}
        \item [(2)]
        \begin{align*}
            \overline{f} + g
            & =
            \overline{f} + f + g\\
            & =
            1 + g\\
            & =
            1
        \end{align*}
        \item [(3)]
        \begin{align*}
            p \cdot \overline{q}
            & =
            \overline{\overline{p} + q}\\
            & =
            \overline{1}\\
            & =
            0
        \end{align*}
    \end{enumerate}    
\end{proof}

\subsection{}   %19
\begin{enumerate}
    \item [(1)]
    \[
        f= x_1 x_2 + x_1 \overline{x}_2 + \overline{x}_1 x_2 + \overline{x}_1 \overline{x}_2 .  
    \]
    \item [(2)]
    \[
        g= x_1 x_2 + \overline{x}_1 \overline{x}_2 .
    \]
\end{enumerate}

\end{document}