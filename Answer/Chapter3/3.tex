\documentclass[UTF8]{ctexart}
\usepackage{verbatim,amsthm,amsfonts,mathdots}
\usepackage{xeCJK,geometry,float,graphicx}
\usepackage{amsmath,amssymb,zhnumber,booktabs,setspace,tasks,booktabs}
\usepackage{tabularray}
\usepackage{cases}
\usepackage{cite}
\usepackage{fancyhdr}
\usepackage{multirow}
\geometry{a4paper}
\pagestyle{fancy}
\fancyhf{}
\setlength{\tabcolsep}{8pt} % Default value: 6pt
\renewcommand{\arraystretch}{2} % Default value: 1
\pagenumbering{arabic}

\begin{document}

\fancyhead[L]{En土土}
\fancyhead[C]{代数结构答案}
\fancyhead[R]{妮可}
\fancyfoot[C]{\thepage}

\section{}
\subsection{}   %1
\begin{enumerate}
    \item [(1)]不能构成映射,如$x_1=0$,则$x_2=0,1,\ldots,9$均满足.
    \item [(2)]能构成映射,$\forall\ y_1\in \mathbb{R},\ \exists !\ y_2 = y_1^2$.
    \item [(3)]不能构成映射,如$y_1 = 1$,则$y_2 = \pm 1$均满足.
\end{enumerate}

\subsection{}   %2
\begin{enumerate}
    \item [(1)]
    \[
        R_f = \{ 2 , -2 , 0 \}    
    \]
    \item [(2)]
    \[
        \begin{cases}
            |A| = 9\\
            |R_f| = 3
        \end{cases}    
        \ \Rightarrow\ 
        n = 3^9.
    \]
\end{enumerate}

\subsection{}   %3
\begin{enumerate}
    \item [(1)]满射
    \[
        \forall\ y\in \mathbb{Z}^{+},\ 
        \exists\ x=\pm(y-1)\in \mathbb{Z},\ 
        f(x)=y.    
    \]
    \item [(2)]既不是单射,又不是满射.值域为$\{0,1,2\}\subseteq \mathbb{Z}\cup\{0\}$
    \[
        \forall\ y\in \mathbb{Z}\cup\{0\}
        \begin{cases}
            y\in \{0,1,2\},\ \exists\ x=3k+y(k\in \mathbb{Z}),\ f(x)=y.\\
            y\notin \{0,1,2\},\ \forall\ x\in \mathbb{Z},\ f(x)\neq y.
        \end{cases}
    \]
    \item [(3)]$f,g$均是双射.
    \[
        \begin{cases}
            \forall\ y\in \mathbb{Z},\ \exists !\ x=y-1\in \mathbb{Z},f(x)=y.\\
            \forall\ y\in \mathbb{Z},\ \exists !\ x=y+1\in \mathbb{Z},g(x)=y. 
        \end{cases}    
    \]
    \item [(4)]满射.
    \[
        \forall\ y\in \{0,1\},\ 
        \exists\ x=2k+y+1(k\in\mathbb{Z}),\ f(x)=y.    
    \]
    \item [(5)]既不是单射,又不是满射.
    \[
        \begin{cases}
            \forall\ y\in \mathbb{Z},\ y<-16,\ \forall\ x\in\mathbb{Z},\ f(x)>y.\\
            \exists\ y=-15,\ f(0)=f(-2)=y.
        \end{cases}
    \]
\end{enumerate}

\subsection{}   %4
\begin{proof}
    \[
        \begin{cases}
            f:A\times B \rightarrow B\times A,\ (a,b) \mapsto (b,a) \\
            g:B\times A \rightarrow A\times B,\ (b,a) \mapsto (a,b) 
        \end{cases}    
    \]
    则
    \[
        f\circ g = {Id}_{A\times B},\quad
        g\circ f = {Id}_{B\times A},\quad
        g=f^{-1}. 
    \]
    即$f$为双射,即证$|A\times B|=|B\times A|$.
\end{proof}

\subsection{}   %5
\begin{proof}
    \begin{enumerate}
        \item []
        \item [(1)]
        \[
            \forall\ f(x)=\sum\limits_{i=0}^{n} a_i\cdot x^i,\ 
            \exists !\ g(x)=\sum\limits_{i=1}^{n} i\cdot a_i \cdot x^{i-1},\ 
            \frac{d}{dx}f(x)=g(x).\ (a_i \in\mathbb{R})
        \]
        值域为$R[x]$,是满射不是双射.
        \[
            \forall\ g(x)=\sum\limits_{i=0}^{n} a_i\cdot x^i,\ 
            \exists\ f(x)=a+\sum\limits_{i=0}^{n} \frac{a_i}{i+1} \cdot x^{i+1},\ 
            \frac{d}{dx}f(x)=g(x).\ (a,a_i \in\mathbb{R})
        \]
        \item [(2)]
        \[
            \forall\ f(x)=\sum\limits_{i=0}^{n} a_i\cdot x^i,\ 
            \exists !\ g(x)=\sum\limits_{i=0}^{n} \frac{a_i}{i+1} \cdot x^{i+1},\ 
            I(f(x))=g(x).\ (a_i \in\mathbb{R})
        \]
        值域为常数项为0的实系数多项式,既不是满射,也不是双射.
        \[
            \forall\ g(x)=a\ (0\neq a\in\mathbb{R}),\ 
            \mbox{若}\exists\ f(x)\in R[x],\mbox{满足} I(f(x))=g(x)=a,
            I(f(x))=g(x).
        \]
        则
        \[
            g(x) = \int_{0}^{x} f(t) \,dt  
            \ \xrightarrow{x=0}\ 
            g(0) = \int_{0}^{0} f(t) \,dt = 0 \neq a.  
        \]
        矛盾.
    \end{enumerate}
\end{proof}

\subsection{}   %6
\begin{proof}
    
\end{proof}

\subsection{}   %7
\begin{proof}
    \begin{enumerate}
        \item []
        \item [(1)]
        \item [(2)]
    \end{enumerate}
\end{proof}

\subsection{}   %8
\begin{enumerate}
    \item [(1)]
    \item [(2)]
\end{enumerate}

\subsection{}   %9
\begin{enumerate}
    \item [(1)]
    \item [(2)]
    \item [(3)]
    \item [(4)]
    \item [(5)]
\end{enumerate}

\subsection{}   %10
\begin{proof}
    
\end{proof}

\subsection{}   %11
\begin{enumerate}
    \item [(1)]
    \item [(2)]
\end{enumerate}

\subsection{}   %12
\begin{enumerate}
    \item [(1)]
    \item [(2)]
    \item [(3)]
    \item [(4)]
\end{enumerate}

\subsection{}   %13
\begin{enumerate}
    \item [(1)]
    \item [(2)]
\end{enumerate}

\subsection{}   %14
\begin{enumerate}
    \item [(1)]
    \item [(2)]
    \item [(3)]
\end{enumerate}

\subsection{}   %15
\begin{enumerate}
    \item [(1)]
    \item [(2)]
\end{enumerate}

\subsection{}   %16
\begin{proof}
    
\end{proof}

\subsection{}   %17
\begin{enumerate}
    \item [(1)]
    \item [(2)]
\end{enumerate}

\subsection{}   %18
\begin{enumerate}
    \item [(1)]
    \item [(2)]
    \item [(3)]
\end{enumerate}

\subsection{}   %19
\begin{enumerate}
    \item [(1)]
    \item [(2)]
\end{enumerate}

\end{document}