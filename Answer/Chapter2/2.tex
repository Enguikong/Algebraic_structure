\documentclass[UTF8]{ctexart}
\usepackage{verbatim,amsthm,amsfonts,mathdots}
\usepackage{xeCJK,geometry,float,graphicx}
\usepackage{amsmath,amssymb,zhnumber,booktabs,setspace,tasks}
\usepackage{cases}
\usepackage{cite}
\usepackage{fancyhdr}
\usepackage{multirow}
\geometry{a4paper}
\pagestyle{fancy}
\fancyhf{}

\pagenumbering{arabic}

\begin{document}

\fancyhead[L]{En土土}
\fancyhead[C]{代数结构答案}
\fancyhead[R]{妮可}
\fancyfoot[C]{\thepage}

\section{数论初步}
\subsection{}   %1
\begin{proof}
    \begin{enumerate}
        \item []
        \item [(1)]
        \[
            \forall\ x|a,\ x|b
            \begin{cases}
                x>0 &\xrightarrow{a>0,x|a}\ x\leq a\\
                x<0 &\xrightarrow{a>0}\ x<a
            \end{cases}    
            \ \Rightarrow\ 
            x<a\ \xrightarrow{a|a,a|b}
            (a,b)=a.
        \]
        \item [(2)]
        \[
            \begin{cases}
                &(a,b)|(a,b),\ (a,b)|b\\
                \\
                &\forall\ x|(a,b),\ x|b,\ \mbox{有}x\leq (a,b).\quad (\mbox{证明同}(1))
            \end{cases}
            \ \Rightarrow\ 
            \left((a,b),b\right)=(a,b).
        \]
    \end{enumerate}
\end{proof}

\subsection{}   %2
\begin{proof}
    \begin{enumerate}
        \item []
        \item [(1)]不妨假设$\exists\ n>0,\ (n,n+1)=d>1$
        \begin{align*}
            (n,n+1)=d\ 
            \Rightarrow\ &
            \exists\ x,y\in \mathbb{Z},\ n=xd,n+1=yd\\
            \Rightarrow\ &
            1=(n+1)-n=(y-x)d>0\\
            \Rightarrow\ &
            y>x,\ (y-x)d\geq d>1\\
            \Rightarrow\ &
            \mbox{矛盾,假设不成立.}
        \end{align*}
        
        \item [(2)]可取$(n,k)$,证明如下
        \[
            \mbox{由推论2.3,取}x=1,\ a=n,\ b=k
            \mbox{,有}(n,k)=(n,n+k).    
        \]
        
    \end{enumerate}
\end{proof}

\subsection{}   %3
\begin{enumerate}
    \item [(1)]$(314,159)=1$,有解。由辗转相除法
    \begin{align*}
        314 & = 159*1 + 155\\
        159 & = 155*1 + 4\\
        155 & = 4*38 + 3\\
        4 & = 3*1 + 1  
    \end{align*}
    即
    \begin{align*}
        1 
        & = 4 - 3*1\\
        & = 4 - (155-4*38)*1\\
        & = (159-155*1)*39 - 155\\
        & = 159*39 - 155*40\\
        & = 159*39 - (314-159*1)*40\\
        & = 159*79 - 314*40
    \end{align*}
    即$x=-40,y=79$.
    \item [(2)]$(3141,1592)=1$,有解。由辗转相除法
    \begin{align*}
        3141 & = 1592*1 + 1549\\
        1592 & = 1549*1 + 43\\
        1549 & = 43*36 + 1
    \end{align*}
    即
    \begin{align*}
        1 
        & = 1549 - 43*36\\
        & = (1592 - 43) - 43*36\\
        & = 1592 - 43*37\\
        & = 1592 - (1592 - 1549*1)*37\\
        & = 1549*37 - 1592*36\\
        & = (3141-1592)*37 - 1592*36\\
        & = 3141*37 - 1592*73
    \end{align*}
    即$x=37,y=-73$.
\end{enumerate}

\subsection{}   %4
\begin{proof}
    
\end{proof}

\subsection{}   %5


\subsection{}   %6


\subsection{}   %7


\subsection{}   %8


\subsection{}   %9


\subsection{}   %10


\subsection{}   %11


\subsection{}   %12


\subsection{}   %13


\subsection{}   %14


\subsection{}   %15


\subsection{}   %16


\subsection{}   %17


\subsection{}   %18


\subsection{}   %19


\subsection{}   %20


\subsection{}   %21


\subsection{}   %22


\subsection{}   %23


\subsection{}   %24


\subsection{}   %25


\subsection{}   %26


\subsection{}   %27


\subsection{}   %28


\subsection{}   %29


\subsection{}   %30


\subsection{}   %31


\subsection{}   %32


\subsection{}   %33


\subsection{}   %34


\subsection{}   %35


\subsection{}   %36


\subsection{}   %37


\subsection{}   %38


\subsection{}   %39


\subsection{}   %40


\subsection{}   %41


\subsection{}   %42    


\end{document}