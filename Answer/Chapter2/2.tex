\documentclass[UTF8]{ctexart}
\usepackage{verbatim,amsthm,amsfonts,mathdots}
\usepackage{xeCJK,geometry,float,graphicx}
\usepackage{amsmath,amssymb,zhnumber,booktabs,setspace,tasks,booktabs}
\usepackage{tabularray}
\usepackage{cases}
\usepackage{cite}
\usepackage{fancyhdr}
\usepackage{multirow}
\geometry{a4paper}
\pagestyle{fancy}
\fancyhf{}
\setlength{\tabcolsep}{8pt} % Default value: 6pt
\renewcommand{\arraystretch}{2} % Default value: 1
\pagenumbering{arabic}

\begin{document}

\fancyhead[L]{En土土}
\fancyhead[C]{代数结构答案}
\fancyhead[R]{妮可}
\fancyfoot[C]{\thepage}

\section{数论初步}
\subsection{}   %1
\begin{proof}
    \begin{enumerate}
        \item []
        \item [(1)]
        \[
            \forall\ x|a,\ x|b
            \begin{cases}
                x>0 &\xrightarrow{a>0,x|a}\ x\leq a\\
                x<0 &\xrightarrow{a>0}\ x<a
            \end{cases}    
            \ \Rightarrow\ 
            x<a\ \xrightarrow{a|a,a|b}
            (a,b)=a.
        \]
        \item [(2)]
        \[
            \begin{cases}
                &(a,b)|(a,b),\ (a,b)|b\\
                \\
                &\forall\ x|(a,b),\ x|b,\ \mbox{有}x\leq (a,b).\quad (\mbox{证明同}(1))
            \end{cases}
            \ \Rightarrow\ 
            \left((a,b),b\right)=(a,b).
        \]
    \end{enumerate}
\end{proof}

\subsection{}   %2
\begin{proof}
    \begin{enumerate}
        \item []
        \item [(1)]不妨假设$\exists\ n>0,\ (n,n+1)=d>1$
        \begin{align*}
            (n,n+1)=d\ 
            \Rightarrow\ &
            \exists\ x,y\in \mathbb{Z},\ n=xd,n+1=yd\\
            \Rightarrow\ &
            1=(n+1)-n=(y-x)d>0\\
            \Rightarrow\ &
            y>x,\ (y-x)d\geq d>1\\
            \Rightarrow\ &
            \mbox{矛盾,假设不成立.}
        \end{align*}
        
        \item [(2)]可取$(n,k)$,证明如下
        \[
            \mbox{由推论2.3,取}x=1,\ a=n,\ b=k
            \mbox{,有}(n,k)=(n,n+k).    
        \]
        
    \end{enumerate}
\end{proof}

\subsection{}   %3
\begin{enumerate}
    \item [(1)]$(314,159)=1$,有解。由辗转相除法
    \begin{align*}
        314 & = 159\cdot 1 + 155\\
        159 & = 155\cdot 1 + 4\\
        155 & = 4\cdot 38 + 3\\
        4 & = 3\cdot 1 + 1  
    \end{align*}
    即
    \begin{align*}
        1 
        & = 4 - 3\cdot 1\\
        & = 4 - (155-4\cdot 38)\cdot 1\\
        & = (159-155\cdot 1)\cdot 39 - 155\\
        & = 159\cdot 39 - 155\cdot 40\\
        & = 159\cdot 39 - (314-159\cdot 1)\cdot 40\\
        & = 159\cdot 79 - 314\cdot 40.
    \end{align*}
    即$x=-40,y=79$.
    \item [(2)]$(3141,1592)=1$,有解。由辗转相除法
    \begin{align*}
        3141 & = 1592\cdot 1 + 1549\\
        1592 & = 1549\cdot 1 + 43\\
        1549 & = 43\cdot 36 + 1
    \end{align*}
    即
    \begin{align*}
        1 
        & = 1549 - 43\cdot 36\\
        & = 1549 - (1592-1549\cdot 1)\cdot 36\\
        & = 1549\cdot 37 - 1592\cdot 36\\
        & = (3141-1592\cdot 1)\cdot 37 - 1592\cdot 36\\
        & = 3141\cdot 37 - 1592\cdot 73.
    \end{align*}
    即$x=37,y=-73$.
\end{enumerate}

\subsection{}   %4
\begin{proof}
    \begin{enumerate}
        \item []
        \item [(0)]$n=1,n^3-n=0$,有$0=6\cdot 0,6|(n^3-n)$.
        \item [(1)]$n=2,n^3-n=0$,有$6=6\cdot 1,6|(n^3-n)$.
        \item [(2)]假设$n=k,k\in \mathbb{N}$时,有$6|(k^3-k)$,则$n=k+1$时有
        \begin{align*}
            {(k+1)}^3 - (k+1)\ 
            = & k^3 + 3k^2 + 2k\\
            = & (k^3-k) + 3k(k+1)
        \end{align*}
        显然有$6|(k^3-k)$,下证$6|3k(k+1)$
        \begin{enumerate}
            \item [$1^\circ$]$k=1,3k(k+1)=6$,有$6=6\cdot 1,6|3k(k+1)$
            \item [$2^\circ$]若$6|3k(k+1)$,则
            \[
                3(k+1)(k+2)=3k(k+1)+6(k+1)  
                \ \Rightarrow\ 
                6|3(k+1)(k+2)
            \]
        \end{enumerate}
        即证
        \[
            \forall\ k\in \mathbb{N},6|3k(k+1)  
            \ \Rightarrow\ 
            6|{(k+1)}^3 - (k+1)
        \]
        综上,即证
        \[
            \forall\ n>0,\ 6|(n^3-n).
        \]
    \end{enumerate}
\end{proof}

\subsection{}   %5
\begin{proof}
    \[
        \begin{cases}
            \ 3^4 \equiv 1(\bmod 10) & \Rightarrow\ 3^{4n} \equiv 1(\bmod 10)\\
            \\
            \ 10 |(3^m+1)   & \Rightarrow\ 3^m \equiv (-1) (\bmod 10)\\
        \end{cases}
        \Rightarrow
        3^{m+4n}\equiv (-1)(\bmod 10).
    \]
    即证 
    \[
        10|(3^{m+4n}+1)
    \]
\end{proof}

\subsection{}   %6
\begin{enumerate}
    \item [(1)]
    \[
        2345 = 5 \cdot  7 \cdot  67
    \]  
    \item [(2)]
    \[
        3456 = 2 \cdot  2 \cdot  2 \cdot  2 \cdot  2 \cdot  2 \cdot  2 \cdot  3 \cdot  3 \cdot  3
    \]
\end{enumerate}

\subsection{}   %7
\begin{proof}
    不妨假设$\exists\ n>0$,使得$n(n+1)=d^2$为平方数,则有
    \[
        n^2 < n(n+1) = d^2 < {(n+1)}^2
        \ \Rightarrow\ 
        n<d<n+1
    \]
    不存在相邻整数间的整数,$d$不存在,假设不成立,即证.
\end{proof}

\subsection{}   %8
\begin{proof}
    \begin{enumerate}
        \item []$n=5!+1=2\cdot 3 \cdot 4 \cdot 5 + 1$.
        \item [(1)]
        \[
            n+1 
            = 2 \cdot 3 \cdot 4 \cdot 5 + 2
            = 2 \cdot (3 \cdot 4 \cdot 5 + 1)
        \]
        \item [(2)]
        \[
            n+2 
            = 2 \cdot 3 \cdot 4 \cdot 5 + 3
            = 3 \cdot (2 \cdot 4 \cdot 5 + 1)
        \]
        \item [(3)]
        \[
            n+3 
            = 2 \cdot 3 \cdot 4 \cdot 5 + 4
            = 4 \cdot (2 \cdot 3 \cdot 5 + 1)
        \]
        \item [(4)]
        \[
            n+4 
            = 2 \cdot 3 \cdot 4 \cdot 5 + 5
            = 5 \cdot (2 \cdot 3 \cdot 4 + 1)
        \]
    \end{enumerate}
\end{proof}

\subsection{}   %9
\begin{enumerate}
    \item [(1)]$(1,1)=1|2$,方程有解,$x_0=0,y_0=2$为一组特解,故通解为
    \[
        \begin{cases}
            \ x & =\  t \quad (t\in \mathbb{Z})\\
            \ y & =\ 2 -t
        \end{cases}
    \]
    \item [(2)]$(2,1)=1|2$,方程有解,$x_0=0,y_0=2$为一组特解,故通解为
    \[
        \begin{cases}
            \ x & =\  t \quad (t\in \mathbb{Z})\\
            \ y & =\ 2 - 2t
        \end{cases}
    \]
    \item [(3)]$(15,16)=1|17$,方程有解,$x_0=-17,y_0=17$为一组特解,故通解为
    \[
        \begin{cases}
            \ x & =\  16t - 17 \quad (t\in \mathbb{Z})\\
            \ y & =\  17 - 15t
        \end{cases}
    \]
\end{enumerate}

\subsection{}   %10
\begin{enumerate}
    \item [(1)]$(6,-15)=3|51$,方程有解,$x_0=11,y_0=1$为一组特解,故通解为
    \[
        \begin{cases}
            \ x & =\  11 - 5t \quad (t\in \mathbb{Z})\\
            \ y & =\  1 - 2t
        \end{cases}
    \]
    又要求负整数解,故$x,y<0,\ t\geq 3$,即所以负整数解为
    \[
        \begin{cases}
            \ x & =\  11 - 5t \quad (t\in \mathbb{Z},\ t\geq 3)\\
            \ y & =\  1 - 2t
        \end{cases}
    \]
    \item [(2)]$(6,15)=3|51$,方程有解,$x_0=6,y_0=1$为一组特解,故通解为
    \[
        \begin{cases}
            \ x & =\  6 + 5t \quad (t\in \mathbb{Z})\\
            \ y & =\  1 - 2t
        \end{cases}
    \]
    又要求负整数解,故$x,y<0,\ t$无解,即无负整数解.
\end{enumerate}

\subsection{}   %11
\subsubsection{必须要用30张}
\begin{enumerate}
    \item []设需要$x$张$5$分,$y$张$1$角,$z=(30-x-y)$张$2$角五分.有
    \begin{align*}
        0.05 x + 0.1 y + 0.25 (30 - x - y) = 5\ 
        &\Leftrightarrow\ 
        x + 2 y + 5 (30 - x - y) = 100 \\
        &\Leftrightarrow\ 
        4x + 3y = 50
    \end{align*}
    \item []$(4,3)=1|50$,方程有解,$x_0=2,y_0=14$为一组特解,故通解为
    \[
        \begin{cases}
            \ x & =\  2 + 3t \quad (t\in \mathbb{Z})\\
            \ y & =\  14 - 4t
        \end{cases}
    \]
    又$x,y,z\in \mathbb{N}$,即
    \[
        \begin{cases}
            \ 2 + 3t  & \geq 0\\
            \ 14 - 4t & \geq 0\\
            \ 14 + t  & \geq 0
        \end{cases}
        \ \xrightarrow{t\in \mathbb{Z}}\ 
        t=0,1,2,3.
    \]
    即有4种方案,记$x$张$5$分,$y$张$1$角,$z$张$2$角五分,则方案为
    \[
        \begin{cases}
            \ x & = 2\\
            \ y & = 14\\
            \ z & = 14
        \end{cases}
        \qquad
        \begin{cases}
            \ x & = 5\\
            \ y & = 10\\
            \ z & = 15
        \end{cases}
        \qquad
        \begin{cases}
            \ x & = 8\\
            \ y & = 6\\
            \ z & = 16
        \end{cases}
        \qquad
        \begin{cases}
            \ x & = 11\\
            \ y & = 2\\
            \ z & = 17 
        \end{cases}
    \]
\end{enumerate}

\subsubsection{不多于30张}
    即求$a,b,c \in \mathbb{N},0.05a+0.1b+0.25c=5$,且$a+b+c\leq 30$.即解
    \[
        \begin{cases}
            \ a+2b+5c = 100 \\
            \ (a+b+c) \leq 30  \\
            \ a,b,c\in \mathbb{N}
        \end{cases}
    \]
    \begin{enumerate}
        \item [(1)]$c\leq 13$
        \[a+2b+5c<2\cdot(30-c)+5c=3c+60\leq 99<100,\mbox{无解.}\]
        \item [(2)]$c >20$
        \[a+2b+5c>5c>100,\mbox{无解.}\]
        \item [(3)]由$(1)(2)$可知$14\leq c \leq 20$
        \[
            a+2b=100-5c,
            (1,2)=1|100-5c
            \Rightarrow
            \mbox{方程存在解}.
        \]
        又$a = 100-5c,\ b=0$为一组特解,故通解为
        \[
            \begin{cases}
                a = 100-5c-2t;\\
                b = t.\ (t\in\mathbb{N})
            \end{cases}
        \]
        \[
            \begin{cases}
                \ a+b+c=100-4k-t &\leq 30 \\
                \ a=100-5c-2t &\geq  0
            \end{cases}
            \Rightarrow 
            \begin{cases}
                \ t\geq 70-4k\\
                \\
                \ t\leq \displaystyle{[\frac{100-5k}{2}]}
            \end{cases}
        \]
        即解的组数为
        \[
            \begin{cases}
                \left[\displaystyle{\frac{100-5k}{2}}\right]-(70-4k)+1; &(70-4k\geq 0)\\
                \\
                \left[\displaystyle{\frac{100-5k}{2}}\right]+1.&(70-4k <0)
            \end{cases}
        \]
        \begin{enumerate}
            \item [(a)]$(c=14)$解的组数为$\left[\displaystyle{\frac{100-5\times 14}{2}}\right]-(70-4\times 14)+1=2$
            \begin{comment}
            \[
                \begin{cases}
                    a&=0;\\
                    b&=15;\\
                    c&=14;
                \end{cases}
                \quad
                \begin{cases}
                    a&=2;\\
                    b&=14;\\
                    c&=14.
                \end{cases}
            \]\end{comment}
            \item [(b)]$(c=15)$解的组数为$\left[\displaystyle{\frac{100-5\times 15}{2}}\right]-(70-4\times 15)+1=3$
            \begin{comment}
            \[
                \begin{cases}
                    a&=1;\\
                    b&=12;\\
                    c&=15;
                \end{cases}
                \quad
                \begin{cases}
                    a&=3 ;\\
                    b&=11 ;\\
                    c&=15 ;
                \end{cases}
                \quad
                \begin{cases}
                    a&=5 ;\\
                    b&=10 ;\\
                    c&=15 .
                \end{cases}
            \]\end{comment}
            \item [(c)]$(c=16)$解的组数为$\left[\displaystyle{\frac{100-5\times 16}{2}}\right]-(70-4\times 16)+1=5$
            \begin{comment}
            \[
                \begin{cases}
                    a&=0 ;\\
                    b&=10 ;\\
                    c&=16 ;
                \end{cases}
                \quad
                \begin{cases}
                    a&=2 ;\\
                    b&=9 ;\\
                    c&=16 ;
                \end{cases}
                \quad
                \begin{cases}
                    a&=4 ;\\
                    b&=8 ;\\
                    c&=16 ;
                \end{cases}
                \quad
                \begin{cases}
                    a&=6 ;\\
                    b&=7 ;\\
                    c&=16 ;
                \end{cases}
                \quad
                \begin{cases}
                    a&=8;\\
                    b&=6;\\
                    c&=16.
                \end{cases}
            \]\end{comment}
            \item [(d)]$(c=17)$解的组数为$\left[\displaystyle{\frac{100-5\times 17}{2}}\right]-(70-4\times 17)+1=6$
            \begin{comment}
            \[
                \begin{cases}
                    a&=1 ;\\
                    b&=7 ;\\
                    c&=17 ;
                \end{cases}
                \quad
                \begin{cases}
                    a&=3 ;\\
                    b&=6 ;\\
                    c&=17 ;
                \end{cases}
                \quad
                \begin{cases}
                    a&=5 ;\\
                    b&=5 ;\\
                    c&=17 ;
                \end{cases}
                \quad
                \begin{cases}
                    a&=7 ;\\
                    b&=4 ;\\
                    c&=17 ;
                \end{cases}
                \quad
                \begin{cases}
                    a&=9 ;\\
                    b&=3 ;\\
                    c&=17 ;
                \end{cases}
                \quad
                \begin{cases}
                    a&=11 ;\\
                    b&=2 ;\\
                    c&=17 ;
                \end{cases}
            \]\end{comment}
            \item [(e)]$(c=18)$解的组数为$\left[\displaystyle{\frac{100-5\times 18}{2}}\right]+1=6$
            \item [(f)]$(c=19)$解的组数为$\left[\displaystyle{\frac{100-5\times 19}{2}}\right]+1=3$
            \item [(g)]$(c=20)$解的组数为$\left[\displaystyle{\frac{100-5\times 20}{2}}\right]+1=1$
        \end{enumerate}
        即共有$2+3+5+6+6+3+1=26$种兑换方法.
    \end{enumerate}

\subsection{}   %12
\begin{enumerate}
    \item []设买了$x$个苹果,$12-x$个橘子,每个苹果$y$分钱,每个橘子$y-3$分钱,则有
    \[
        \begin{cases}
            \ 0 \leq 12-x  < x \\
            \ xy + (12-x)(y-3)  = 99
        \end{cases}
        \ \Leftrightarrow\
        \begin{cases}
            \ 6 < x \leq 12 \\
            \ x + 4y = 45
        \end{cases} 
    \]
    $(1,4)=1|45$,方程有解,$x_0=9,y_0=9$为一组特解,故通解为
    \[
        \begin{cases}
            \ x & =\  9 + 4t \quad (t\in \mathbb{Z})\\
            \ y & =\  9 - t
        \end{cases}
    \]
    又$6 < x \leq 12 $,即$t=0,\ x=9,12-x=3$,买了9个苹果和3个橘子.
\end{enumerate}

\subsection{}   %13
\[
    6k+5 \equiv 6k+1 (\bmod 4) 
\]
又$6k \equiv 6 (\bmod 4)$,有
\[
    \begin{aligned}
        6k+5 
        &\equiv 7 (\bmod 4)\\
        &\equiv 3 (\bmod 4)
    \end{aligned}
\]

\subsection{}   %14
\begin{proof}
    \begin{enumerate}
        \item []
        \item [(1)]分情况讨论$6k,6k+2,6k+3,6k+4\ (k\geq 1)$即可,不再赘述.
        \item [(2)]记素数为$p,p>3$.
        \begin{enumerate}
            \item [(a)]$p<6$,则$p=5$,成立.
            \item [(b)]$p>6$,有$(6,p)=1$,故$p$属于6的缩系,故$p$模$6$或与1或5同余.
        \end{enumerate}
    \end{enumerate}
\end{proof}

\subsection{}   %15
\begin{proof}
    \begin{enumerate}
        \item []
        \item []不妨设这两个连续的立方数为$k^3$与${(k+1)}^3$.
        \[
            \begin{aligned}
                {(k+1)}^3 - k^3 
                \equiv & 3k^2 + 3k + 1 (\bmod 3) \\
                \equiv & 1 (\bmod 3)
            \end{aligned}
        \]
    \end{enumerate}
\end{proof}

\subsection{}   %16
\begin{proof}
    \begin{enumerate}
        \item []
        \item []设该数为$A=\overline{a_n a_{n-1} \ldots a_1 a_0}$,则
        \[
            A=\sum\limits_{i=0}^{i=n} a_i\cdot 10^{i},\quad
            \sum\limits_{i=0}^{i=n} a_i \equiv 0 (\bmod 3)
        \]
        又$\forall\ k\in \mathbb{N},\ 10^{i} \equiv 1 (\bmod 3)$,故
        \begin{align*}
            A 
            \equiv & \sum\limits_{i=0}^{i=n} a_i\cdot 10^{i} (\bmod 3)\\
            \equiv & \sum\limits_{i=0}^{i=n} a_i(\bmod 3)\\
            \equiv & 0(\bmod 3)
        \end{align*}
    \end{enumerate}
\end{proof}

\subsection{}   %17
\begin{proof}
    \begin{enumerate}
        \item []
        \item [(1)]
        \[
            10 \equiv -1 (\bmod 11)
            \ \Rightarrow\ 
            10^k \equiv {(-1)}^{k} (\bmod 11)
        \]
        \item [(2)]设数为$A=\overline{a_n a_{n-1} \ldots a_1 a_0}$,则
        \[
            A \equiv 0 (\bmod 11)
            \ \Leftrightarrow\ 
            \sum\limits_{i=0}^{n} {(-1)}^{i} \cdot a_i 
            \equiv 0 (\bmod 11)
        \]
        即偶数位之和与奇数位之和的差能被$11$整除等价于该数也能被$11$整除.
    \end{enumerate}
\end{proof}

\subsection{}   %18
\begin{enumerate}
    \item [(1)]
    \[
        \begin{aligned}
            2x
            \equiv & 1 (\bmod 17)\\
            \equiv & 18 (\bmod 17)
        \end{aligned}      
        \ \xrightarrow{(2,17)=1}\ 
        x\equiv 9 (\bmod 17)
    \]
    \item [(2)]$(3,18)=3|6$,故有3组解
    由$x\equiv 2 (\bmod 6)$得原方程解为
    \[
        x\equiv 2 + 6t (\bmod 18)
        \quad (0\leq t \leq 2).
    \]
    即
    \[
        x\equiv 2,8,14 (\bmod 18)  .
    \]
    \item [(3)]$(4,18)=2|6$,故有2组解
    解$2x\equiv 3 (\bmod 9)$
    \[
        \begin{aligned}
            2x
            \equiv & 3 (\bmod 9)\\
            \equiv & 12 (\bmod 9)
        \end{aligned}
        \ \xrightarrow{(2,9)=1}\ 
        x\equiv 6(\bmod 9).
    \]
    即原方程解为
    \[
        x\equiv 6+9t (\bmod 18)\quad (t=0,1)
        \ \Rightarrow\ 
        x\equiv 6,15 (\bmod 18)  .
    \]
    
    \item [(4)]
    \[
        \begin{aligned}
            3x 
            \equiv & 1 (\bmod 17)\\
            \equiv & 18 (\bmod 17)
        \end{aligned}   
        \ \xrightarrow{(3,17)=1}\ 
        x\equiv 6(\bmod 17) .
    \]
    
\end{enumerate}

\subsection{}   %19
\begin{enumerate}
    \item [(1)]$(2,3)=1$,有解。
    本题中
    \[
        M = 2\cdot 3 = 6  , M_1 = 3 , M_2 = 2.
    \]
    由
    \[
        \begin{cases}
            \ 3 b_1 &\equiv 1 (\bmod 2)\\
            \ 2 b_2 &\equiv 1 (\bmod 3)
        \end{cases}  
        \ \Rightarrow\ 
        \begin{cases}
            \ b_1 & = 1 \\
            \ b_2 & = 2 
        \end{cases}
    \]
    从而
    \[
        \begin{aligned}
            y 
            & = 3\cdot 1 \cdot 1 + 2\cdot 1 \cdot 2 \\
            & = 7
        \end{aligned}
        \ \Rightarrow\ 
        y\equiv 1 (\bmod 6).
    \]
    \item [(2)]$(41,26)=1$,有解。
    原式等价于
    \[
        \begin{cases}
            \ x \equiv & 31 (\bmod 41)\\
            \ x \equiv & 7 (\bmod 26)
        \end{cases}
    \]
    本题中
    \[
        M = 41\cdot 26 , M_1 = 26 , M_2 = 41.  
    \]
    由
    \[
        \begin{cases}
            \ 26 b_1 & \equiv 1 (\bmod 41)\\
            \ 41 b_2 & \equiv 1 (\bmod 26)
        \end{cases}  
        \ \Rightarrow\ 
        \begin{cases}
            \ b_1 & = 30\\
            \ b_2 & = 7
        \end{cases}
    \]
    从而
    \[
        \begin{aligned}
            y 
            & = 26\cdot 31 \cdot 30 + 41\cdot 7 \cdot 7 \\
            & = 26819
        \end{aligned}
        \ \Rightarrow\ 
        y\equiv 605 (\bmod 1066).
    \]
    \item [(3)]$(2,3)=(2,7)=(3,7)=1$,有解。
    本题中
    \[
        M = 2\cdot 3\cdot 7 =42 , M_1 = 21  , M_2 = 14, M_3 = 6.  
    \]
    由
    \[
        \begin{cases}
            \ 21 b_1 & \equiv 1 (\bmod 2)\\ 
            \ 14 b_2 & \equiv 1 (\bmod 3)\\
            \ 6  b_3 & \equiv 1 (\bmod 7)
        \end{cases}  
        \ \Rightarrow\ 
        \begin{cases}
            \ b_1 & = 1 \\
            \ b_2 & = 2 \\
            \ b_3 & = 6
        \end{cases}
    \]
    从而
    \[
        \begin{aligned}
            y 
            & = 21\cdot 1 \cdot 1 + 14\cdot 1 \cdot 2 + 6\cdot 6\cdot 6 \\
            & = 265 . 
        \end{aligned}
        \ \Rightarrow\ 
        y\equiv 13 (\bmod 42).
    \]
    \item [(4)]原式等价于
    \[
        \begin{cases}
            \ x & \equiv 3 (\bmod 5)\\
            \ x & \equiv 3 (\bmod 7)\\
            \ x & \equiv 3 (\bmod 11)
        \end{cases}  
    \]
    $(5,7)=(5,11)=(7,11)=1$,有解。
    本题中
    \[
        M = 5\cdot 7\cdot 11 = 385 , M_1 = 77 , M_2 = 55 , M_3 = 35
    \]
    由
    \[
        \begin{cases}
            \ 77 b_1 \equiv & 1 (\bmod 5)   \\
            \ 55 b_2 \equiv & 1 (\bmod 7)   \\
            \ 35 b_3 \equiv & 1 (\bmod 11)
        \end{cases}  
        \ \Rightarrow\ 
        \begin{cases}
            \ b_1 & = 3\\ 
            \ b_2 & = 6\\
            \ b_3 & = 6
        \end{cases}
    \]
    从而
    \[
        \begin{aligned}
            y 
            & = 77\cdot 3 \cdot 3 + 55\cdot 3 \cdot 6 + 35\cdot 3\cdot 6 \\
            & = 2313 . 
        \end{aligned}
        \ \Rightarrow\ 
        y\equiv 3 (\bmod 385).
    \]
\end{enumerate}

\subsection{}   %20
\begin{enumerate}
    \item []设
    \[
        \begin{cases}
            \ 3x\equiv m-1 &\pmod{20}\\
            \ 5y\equiv m &\pmod{20}\\
            \ 7z\equiv m+1 &\pmod{20}.
        \end{cases}
        \qquad (1\leq m \leq 18)
    \]    
    \item []则
    \[
        \begin{cases}
            \ 3x&=20(m-1)+(m-1)\\
            \ 5y&=20m+m\\
            \ 7z&=20(m+1)+(m+1).
        \end{cases}
        \Rightarrow
        \begin{cases}
            \ x&=7m-7\\
            \ y&=\displaystyle{\frac{21m}{5}}\\
            \ z&=3m+3.
        \end{cases}
        \Rightarrow
        5|m,m\in\{5,10,15\}
    \]
    \item []即
    \[
        \begin{cases}
            \ x&=28\\
            \ y&=21\\
            \ z&=18;
        \end{cases}
        \quad
        \begin{cases}
            \ x&=63\\
            \ y&=42\\
            \ z&=33;
        \end{cases}
        \quad
        \begin{cases}
            \ x&=98\\
            \ y&=63\\
            \ z&=48.
        \end{cases}
    \]
    
\end{enumerate}

\subsection{}   %21
\begin{enumerate}
    \item []由题意有
    \[
        \begin{cases}
            \ n   &\equiv 0 \pmod 2\\
            \ n+1 &\equiv 0 \pmod 3\\
            \ n+2 &\equiv 0 \pmod 4\\
            \ n+3 &\equiv 0 \pmod 5\\
            \ n+4 &\equiv 0 \pmod 6
        \end{cases}
        \ \Leftrightarrow\ 
        \begin{cases}
            \ n &\equiv 0 \pmod 2\\
            \ n &\equiv 2 \pmod 3\\
            \ n &\equiv 2 \pmod 4\\
            \ n &\equiv 2 \pmod 5\\
            \ n &\equiv 2 \pmod 6
        \end{cases}
    \]
    \item []由$n=2$为一个特解,有模$[2,3,4,5,6]=60$唯一解
    \[
        n \equiv 2 \pmod {60}
    \]
    \item []故所求最小整数$n(n>2)$为
    \[
        n=62.
    \]
\end{enumerate}

\subsection{}   %22
\begin{enumerate}
    \item [(1)]
    \[
       \phi (42) 
       = \phi (2 \cdot 3 \cdot 7) 
       = \phi (2) \cdot \phi (3) \cdot \phi (7)
       = 1\cdot 2 \cdot 6
       = 12.
    \]
    \item [(2)]
    \[
        \phi (420) 
       = \phi (2^2 \cdot 3 \cdot 5 \cdot 7) 
       = \phi (2^2) \cdot \phi (3) \cdot \phi (5) \cdot \phi (7)
       = 2\cdot 2 \cdot 4\cdot 6
       = 96.
    \]
    \item [(3)]
    \[
        \phi (4200) 
       = \phi (2^3 \cdot 3 \cdot 5^2 \cdot 7) 
       = \phi (2^3) \cdot \phi (3) \cdot \phi (5^2) \cdot \phi (7)
       = 4\cdot 2 \cdot 20\cdot 6
       = 960.
    \]
\end{enumerate}

\subsection{}   %23
\begin{enumerate}
    \item [(1)]小于18且与18互素的正整数有
    \[
        1,5,7,11,13,17  
    \]
    \item [(2)]
    \[
        \begin{aligned}
            1\cdot 5 
            & \equiv 5 (\bmod 18)\\
            &
        \end{aligned}    
        \qquad \qquad
        \begin{aligned}
            5\cdot 5 
            & \equiv 25 (\bmod 18)\\
            & \equiv 7 (\bmod 18)
        \end{aligned}
    \]
    \[
        \begin{aligned}
            7\cdot 5
            & \equiv 35 (\bmod 18)\\
            & \equiv 17 (\bmod 18)
        \end{aligned}
        \qquad \qquad
        \begin{aligned}
            11\cdot 5 
            & \equiv 55 (\bmod 18)\\
            & \equiv 1 (\bmod 18)
        \end{aligned}  
    \]
    \[
        \begin{aligned}
            13\cdot 5 
            & \equiv 65 (\bmod 18)\\
            & \equiv 11 (\bmod 18)
        \end{aligned}
        \qquad \qquad
        \begin{aligned}
            17\cdot 5
            & \equiv 85 (\bmod 18)\\
            & \equiv 13 (\bmod 18)
        \end{aligned}
    \]
    仍为缩系,引理$2.1$成立.
\end{enumerate}

\subsection{}   %24
设$m,n$有素数分解
\[
    m = m_1^{k_1} m_2^{k_2} \cdots m_x^{k_x} \cdot p^{M},
    \quad
    n = n_1^{l_1} n_2^{l_2} \cdots n_y^{l_y} \cdot p^{N}
\]
且
\[
    \forall\ 1\leq i \leq x, 1\leq j \leq y,
    \mbox{有}
    m_i \neq n_j.
    \qquad
    (m_i,n_j\mbox{均为素数})
\]
\begin{align*}
    \phi (mn) 
    & = 
    \phi \left( p^{M+N}
    \cdot \prod\limits_{i=1}^{x} m_i^{k_i} 
    \cdot \prod\limits_{j=1}^{y} n_{j}^{l_{j}} \right)\\
    & = 
    \phi (p^{M+N}) 
    \cdot \prod\limits_{i=1}^{x} \phi (m_i^{k_i}) 
    \cdot \prod\limits_{j=1}^{y} \phi (n_{j}^{l_{j}})\\
    & = 
    p^{M+N}\cdot \displaystyle{(1-\frac{1}{p})}
    \cdot \prod\limits_{i=1}^{x} m_i^{k_i} \displaystyle{(1-\frac{1}{m_i})}
    \cdot \prod\limits_{j=1}^{y} n_{j}^{l_{j}} \displaystyle{(1-\frac{1}{n_j})}\\
    & =
    mn \cdot \displaystyle{(1-\frac{1}{p})}
    \cdot \prod\limits_{i=1}^{x} \displaystyle{(1-\frac{1}{m_i})}
    \cdot \prod\limits_{j=1}^{y} \displaystyle{(1-\frac{1}{n_j})}
\end{align*}
\begin{align*}
    \phi (m) \phi(n) 
    & = 
    \phi \left( p^{M}
        \cdot \prod\limits_{i=1}^{x} m_i^{k_i} 
    \right)
    \cdot
    \phi \left( p^{N}
        \cdot \prod\limits_{j=1}^{y} n_{j}^{l_{j}}
    \right)
    \\
    & = 
    \phi (p^{M}) \cdot \phi (p^{N})
    \cdot \prod\limits_{i=1}^{x} \phi (m_i^{k_i}) 
    \cdot \prod\limits_{j=1}^{y} \phi (n_{j}^{l_{j}})\\
    & = 
    p^{M}\cdot \displaystyle{(1-\frac{1}{p})}
    \cdot p^{N}\cdot \displaystyle{(1-\frac{1}{p})}
    \cdot \prod\limits_{i=1}^{x} m_i^{k_i} \displaystyle{(1-\frac{1}{m_i})}
    \cdot \prod\limits_{j=1}^{y} n_{j}^{l_{j}} \displaystyle{(1-\frac{1}{n_j})}\\
    & =
    mn \cdot \displaystyle{{(1-\frac{1}{p})}^2}
    \cdot \prod\limits_{i=1}^{x} \displaystyle{(1-\frac{1}{m_i})}
    \cdot \prod\limits_{j=1}^{y} \displaystyle{(1-\frac{1}{n_j})}
\end{align*}
即
\[
    \phi (m) \phi(n) = \displaystyle{(1-\frac{1}{p})} \cdot \phi (mn)
\]

\subsection{}   %25
\begin{proof}
    \begin{enumerate}
        \item []
        \item []显然有$n\geq 0$,否则$\phi (n) \geq 0 > n$,问题无意义.
        \item [(1)]$6|n$即$2|n,3|n$,不妨记$n$有素数分解
        \[
            n=2^{p}\cdot 3^{q} \cdot n_1^{k_1} n_2^{k_2} \cdots n_N^{k_N}.
            \quad
            (p,q \geq 1)
        \]
        则
        \begin{align*}
            \phi (n) 
            & = \phi (2^{p} \cdot 3^{q}\cdot n_1^{k_1} n_2^{k_2} \cdots n_N^{k_N}) \\
            & = \phi (2^{p})\cdot \phi (3^{q}) \cdot \prod\limits_{i=1}^{N} \phi (n_i^{k_i})\\
            & = n \cdot \displaystyle{\left(1-\frac{1}{2} \right) \cdot \left(1-\frac{1}{3}\right)\cdot \prod_{i=1}^{N} \left(1-\frac{1}{n_i}\right) }\\
            & = \displaystyle{\frac{n}{3} \cdot \prod_{i=1}^{N} \left(1-\frac{1}{n_i}\right)}\\
            & \leq \displaystyle{\frac{n}{3}}
        \end{align*}
        即证,且当且仅当$N=0$时等号成立.
        \item [(2)]由$T14$可知
        \[
            n-1 \equiv 5 \pmod{6} ; 
            \qquad
            n+1 \equiv 1 \pmod{6} ;
            \qquad
            n \equiv 0 \pmod{6}
        \]
        即$6|n$,由$(1)$即证.
    \end{enumerate}
\end{proof}

\subsection{}   %26
\begin{enumerate}
    \item [(1)]
    \[
        \frac{3}{2} \phi(3) 
            = \frac{3}{2}\cdot 2  
            = 3 
            = 1 + 2. 
        \qquad
        \frac{4}{2} \phi(4)
            = \frac{4}{2} \cdot 2 \\
            = 4 \\
            = 1+3 .
    \]
    \[
        \frac{5}{2}\phi(5) = \frac{5}{2}\cdot 4 = 10 = 1+2+3+4.
        \qquad
        \frac{6}{2}\phi(6) = \frac{6}{2}\cdot 2 = 6 =1+5.
    \]
    \[
        \frac{7}{2}\phi(7) = \frac{7}{2}\cdot 6 = 21= 1+2+3+4+5+6.
        \qquad
        \frac{8}{2}\phi(8) = \frac{8}{2}\cdot 4 = 16 =1+3+5+7.
    \]
    \item [(2)]对于整数$n\geq 3$,缩系中所有数的和等于$\frac{1}{2} n \cdot \phi (n)$.即
    \[
        \sum\limits_{\substack{ (d,n)=1\\ 1\leq d\leq n-1} } d
        = \frac{1}{2}\phi(n) \cdot n.
    \]
    \item [(3)]
    \begin{proof}
        \begin{enumerate}
            \item []
            \item []$\forall\ 1 \leq d \leq n-1,\ (d,n)=1$,有$(n-d,n)=1$,且$d\neq n/2$,故
            \[
                \sum\limits_{\substack{ (d,n)=1\\ 1\leq d\leq n-1} } d
                =
                \sum\limits_{\substack{ (d,n)=1\\ 1\leq d < n/2 } } d + (n-d)
                =
                \frac{1}{2} n \cdot \phi(n)
            \]
        \end{enumerate}
         
    \end{proof}
\end{enumerate}

\subsection{}   %27
\begin{align*}
    {314}^{159} 
    & \equiv {(7\cdot 45 - 1)}^{159} \pmod{7}\\
    & \equiv {(-1)}^{159} \pmod{7}\\
    & \equiv (-1) \pmod{7}\\
    & \equiv 6 \pmod{7}
\end{align*}

\subsection{}   %28
\begin{enumerate}
    \item [(1)]求末位即求模$10$余数
    \begin{align*}
        7^{355} 
        & \equiv {(7^{4})}^{88} \cdot 7^{3} \pmod{10}\\
        & \equiv {(2400 + 1)}^{88} \cdot {(340 + 3)} \pmod{10}\\
        & \equiv 3 \pmod{10}
    \end{align*}
    即末位为$3$.用欧拉定理$7^{\phi(10)}\equiv 1\pmod{10}$亦可.
    \item [(2)]求末两位即求模$100$余数
    \begin{align*}
        7^{355}
        & \equiv {(7^{4})}^{88} \cdot 7^{3} \pmod{100}\\
        & \equiv {(2400 + 1)}^{88} \cdot {(300 + 43)} \pmod{100}\\
        & \equiv 43 \pmod{100}
    \end{align*}
    即末两位为$43$.用欧拉定理$7^{\phi(100)}\equiv 1\pmod{100}$亦可.
\end{enumerate}

\subsection{}   %29
\begin{proof}
    \begin{enumerate}
        \item []
        \item [(1)]
        \begin{align*}
            {(k+1)}^p - k^p \equiv 1 \pmod{p}\ 
            & \Leftrightarrow\ 
            p\mid {(k+1)}^p - k^p -1 \\
            & \Leftrightarrow\ 
            p\mid \sum\limits_{i=1}^{p-1} C_{p}^{i}\cdot k^{p-i}
        \end{align*}
        有
        \[
            C_{p}^{i}=
            \displaystyle{
                \frac{p(p-1)\ldots (p-i+1)}{i!}
            }
            \in \mathbb{N},
            \ \Rightarrow\ 
            i! \mid p(p-1)\ldots (p-i+1).
            \quad (1\leq i \leq p-1)
        \]
        又$\forall\ 1\leq i\leq p-1,\ (p,i)=1$,故$(p,i!)=1$,即
        \[
            i!\mid (p-1)\ldots (p-i+1)
            \ \Rightarrow\ 
            \displaystyle{
                \frac{(p-1)\ldots(p-i+1)}{i!}
            }
            \in \mathbb{N},\ 
            p\mid C_{p}^{i}.
        \]
        故有
        \[
            p\mid C_{p}^{i}
            \ \Rightarrow\ 
            p\mid \sum\limits_{i=1}^{p-1} C_{p}^{i}\cdot k^{p-i}
            \ \Rightarrow\ 
            {(k+1)}^p - k^p \equiv 1 \pmod{p}.
        \]
        \item [(2)]对于任意素数$p$有$p\nmid a$,则
        \[
            \begin{aligned}
                a^{p}
                & \equiv 
                \sum\limits_{k=0}^{a-1} \left({(k+1)}^{p} - {k}^{p} \right) \pmod{p}\\
                & \equiv 
                \sum\limits_{k=0}^{a-1} 1 \pmod{p}\\
                & \equiv
                a \pmod{p}
            \end{aligned}
        \]
        又$(a^p,a)=a,(p,a)=1$,故有
        \[
            a^{p-1} \equiv 1 \pmod{p}.
        \]
    \end{enumerate}
\end{proof}

\subsection{}   %30
\begin{proof}
    \begin{enumerate}
        \item []
        \item [(1)]
        \[
            \forall\ 1\leq k \leq p-1,\ 
            (k,p)=1\mbox{且}p\mid k
            \ \Rightarrow\ 
            k^{p-1} \equiv 1 \pmod{p}
        \]
        故
        \begin{align*}
            \sum\limits_{i=1}^{p-1} i^{p-1}
            & \equiv 
            \sum\limits_{i=1}^{p-1} 1 \pmod{p}\\
            & \equiv
            p-1 \pmod{p}\\
            & \equiv
            -1 \pmod{p}.
        \end{align*}
        \item [(2)]
        \begin{align*}
            \forall\ 1\leq k \leq p-1,\ 
            (k,p)=1\mbox{且}p\mid k
            & \Rightarrow\ 
            k^{p-1} \equiv 1 \pmod{p}\\
            & \Rightarrow\ 
            k^{p} \equiv k \pmod{p}
        \end{align*}
        故
        \begin{align*}
            \sum\limits_{i=1}^{p-1} i^{p-1}
            & \equiv 
            \sum\limits_{i=1}^{p-1} i \pmod{p}\\
            & \equiv
            \displaystyle{\frac{p(p-1)}{2}} \pmod{p}\\
            & \equiv
            0 \pmod{p}.\qquad (2|p-1)
        \end{align*}
    \end{enumerate}
\end{proof}

\subsection{}   %31
\[
    \begin{aligned}
        d(42)
        & = 
        d(2\cdot 3 \cdot 7)\\
        & =
        2^3\\
        & = 
        8.
    \end{aligned}
    \qquad\quad
    \begin{aligned}
        d(420)
        & = 
        d(2^2 \cdot 3 \cdot 5 \cdot 7)\\
        & =
        3\cdot 2^3 \\
        & = 
        24.
    \end{aligned}
    \qquad\quad
    \begin{aligned}
        d(4200)
        & = 
        d(2^3 \cdot 3 \cdot 5^2 \cdot 7)\\
        & =
        4\cdot 3\cdot 2^2 \\
        & = 
        48.
    \end{aligned}
\]
\begin{align*}
    &\begin{aligned}
        \sigma(42)
        & = 
        \sigma(2\cdot 3 \cdot 7)\\
        & =
        \displaystyle{
            \frac{2^2 -1 }{2-1}\cdot 
            \frac{3^2 -1 }{3-1}\cdot
            \frac{7^2 -1 }{7-1}
        }\\
        & = 
        96.
    \end{aligned} 
    \qquad
    \qquad
    \begin{aligned}
        \sigma(420)
        & = 
        \sigma(2^2 \cdot 3 \cdot 5 \cdot 7)\\
        & =
        \displaystyle{
            \frac{2^3 -1}{2-1}\cdot 
            \frac{3^2 -1}{3-1}\cdot
            \frac{5^2 -1}{5-1}\cdot
            \frac{7^2 -1}{7-1}
        }\\
        & = 
        1344.
    \end{aligned} \\
    \\
    &\begin{aligned}
            \sigma(4200)
            & = 
            \sigma(2^3 \cdot 3 \cdot 5^2 \cdot 7)\\
            & =
            \displaystyle{
                \frac{2^4 - 1}{2-1}\cdot 
                \frac{3^2 - 1}{3-1}\cdot
                \frac{5^3 - 1}{5-1}\cdot
                \frac{7^2 - 1}{7-1}
            }\\
            & = 
            14880.
        \end{aligned}
\end{align*}

\subsection{}   %32
不妨设$n$有素数分解
\[
    n = n_1^{k_1} n_2^{k_2} \cdots n_x^{k_x}
    \ \Rightarrow\ 
    \sigma(n) = 
    (k_1 + 1) (k_2 + 1 )\ldots (k_x + 1)=60
\]
又
\[
    \left\lceil \log_{2} (10^4)\right\rceil = 13  
    \ \Rightarrow\ 
    k_1 + k_2 + \cdots + k_x \leq 13  
\]
\[
    n = 2^4 \cdot 3^2 \cdot 5\cdot 7 = 5040
    \qquad
    \mbox{或}
    \qquad
    n = 2^4 \cdot 5^2 \cdot 3\cdot 7 = 8400
\]

\subsection{}   %33
\begin{proof}
    \begin{enumerate}
        \item []
        \item []设$d_1,d_2,\ldots,d_k$为$n$的全部因子 (相同因子算两遍),则
        \[
            \forall\ 1\leq i \leq k,\ \exists!\ 1 \leq j \leq k,\ 
            d_j = n/d_i
        \]
        不妨取一个排列使得$i+j=k+1$.
        \begin{align*}
            \sum\limits_{d\mid n} \frac{1}{d}
            & =
            \frac{1}{n} \sum\limits_{i=1}^{k} \frac{n}{d_i}\\
            & = 
            \frac{1}{n} \sum\limits_{i=1}^{k} d_{k+1-i}\\
            & =
            \frac{1}{n} \sum\limits_{i=1}^{k} d_{i}\\
            & = 
            \frac{1}{n} \sigma(n).
        \end{align*}
    \end{enumerate}
\end{proof}

\subsection{}   %34
\begin{proof}
    \begin{enumerate}
        \item []
        \item []不妨记偶完全数为
        \[
            n =2^{p-1}\cdot (2^p - 1)    
            \qquad
            (p,2^{p}-1 \mbox{均为素数})
        \]
        由题意可得
        \[
            2^{p-1}\cdot (2^p - 1) \equiv 6 \pmod{10}
            \quad
            \mbox{或}    
            \quad
            2^{p-1}\cdot (2^p - 1) \equiv 8 \pmod{10}
        \]
        等价于
        \[
            2^{p-2}\cdot (2^p - 1) \equiv 3 \pmod{5}
            \quad
            \mbox{或}    
            \quad
            2^{p-2}\cdot (2^p - 1) \equiv 4 \pmod{5}
        \]
        \[
            \begin{cases}
                (2^{p-2} , 5)  &= 1\\
                (2^{p} -1 , 5) &=1
            \end{cases}
            \ \Rightarrow\ 
            \begin{cases}
                2^{p-2} & \equiv a \pmod{5} \quad(a\in \{1,2,3,4\})\\
                2^{p}-1 & \equiv b \pmod{5} \quad(b\in \{1,2,3,4\})
            \end{cases}
        \]
        \item [(1)]$2^{p-2}\equiv 1 \pmod{5}$
        \[
            \begin{aligned}
                2^{p} - 1 
                & \equiv  
                2^2\cdot 1  - 1 \pmod{5}\\
                & \equiv
                3 \pmod{5} 
            \end{aligned}
            \quad \Rightarrow\quad 
            \begin{aligned}
                2^{p-2}\cdot (2^p - 1) 
                & \equiv
                3\cdot 1 \pmod{5}\\
                & \equiv
                3 \pmod{5}
            \end{aligned}
        \]
        \item [(2)]$2^{p-2}\equiv 2 \pmod{5}$
        \[
            \begin{aligned}
                2^{p} - 1 
                & \equiv  
                2^2\cdot 2  - 1 \pmod{5}\\
                & \equiv
                2 \pmod{5} 
            \end{aligned}
            \quad \Rightarrow\quad 
            \begin{aligned}
                2^{p-2}\cdot (2^p - 1) 
                & \equiv
                2\cdot 2 \pmod{5}\\
                & \equiv
                4 \pmod{5}
            \end{aligned}
        \]
        \item [(3)]$2^{p-2}\equiv 3 \pmod{5}$
        \[
            \begin{aligned}
                2^{p} - 1 
                & \equiv  
                2^2\cdot 3  - 1 \pmod{5}\\
                & \equiv
                1 \pmod{5} 
            \end{aligned}
            \quad \Rightarrow\quad 
            \begin{aligned}
                2^{p-2}\cdot (2^p - 1) 
                & \equiv
                3\cdot 1 \pmod{5}\\
                & \equiv
                3 \pmod{5}
            \end{aligned}
        \]
        \item [(4)]$2^{p-2}\equiv 4 \pmod{5}$
        \[
            \begin{aligned}
                2^{p} - 1 
                & \equiv  
                2^2\cdot 4  - 1 \pmod{5}\\
                & \equiv
                0 \pmod{5} 
            \end{aligned}
            \quad \Rightarrow\quad 
            0\notin \{1,2,3,4\},
            \mbox{该情况不存在}.
        \]
        综上,即证
        \[
            2^{p-1}\cdot (2^p - 1) \equiv 6 \pmod{10}
            \quad
            \mbox{或}    
            \quad
            2^{p-1}\cdot (2^p - 1) \equiv 8 \pmod{10}
        \]
    \end{enumerate}
\end{proof}

\subsection{}   %35
\begin{proof}
    \begin{enumerate}
        \item []
        \item []由题意可得
        \[
            n =2^{p-1}\cdot (2^p - 1)    
            \qquad
            (p,\ 2^{p}-1 \mbox{均为素数})
        \]
        又$n>6$,故$p>2,\ 2|p-1$,不妨记$p-1=2k\ (k\in \mathbb{Z}^{*})$,有
        \[
            n = 2^{2k} \cdot (2^{2k+1} - 1)
            = 4^{k} \cdot (2\cdot 4^{k} -1)    
        \]
        又
        \[
            \forall\ k\geq 1,\ 
            4^{k} \equiv m \pmod{9},\ 
            \mbox{有}
            m\in\{4,7, 1\}
        \]
        \item [(1)]$4^{k} \equiv 4 \pmod{9}$
        \[
            \begin{aligned}
                2\cdot 4^{k} - 1
                & \equiv 
                2\cdot 4 -1 \pmod{9}\\
                & \equiv
                7 \pmod{9}
            \end{aligned}
            \quad \Rightarrow \quad
            \begin{aligned}
                4^{k}\cdot (2\cdot 4^{k} -1)
                & \equiv
                4\cdot 7 \pmod{9}\\
                & \equiv
                1 \pmod{9}
            \end{aligned}  
        \]
        \item [(2)]$4^{k} \equiv 7 \pmod{9}$
        \[
            \begin{aligned}
                2\cdot 4^{k} - 1
                & \equiv 
                2\cdot 7 -1 \pmod{9}\\
                & \equiv
                4 \pmod{9}
            \end{aligned}
            \quad \Rightarrow \quad
            \begin{aligned}
                4^{k}\cdot (2\cdot 4^{k} -1)
                & \equiv
                7\cdot 4 \pmod{9}\\
                & \equiv
                1 \pmod{9}
            \end{aligned}  
        \]
        \item [(3)]$4^{k} \equiv 1 \pmod{9}$
        \[
            \begin{aligned}
                2\cdot 4^{k} - 1
                & \equiv 
                2\cdot 1 -1 \pmod{9}\\
                & \equiv
                1 \pmod{9}
            \end{aligned}
            \quad \Rightarrow \quad
            \begin{aligned}
                4^{k}\cdot (2\cdot 4^{k} -1)
                & \equiv
                1\cdot 1 \pmod{9}\\
                & \equiv
                1 \pmod{9}
            \end{aligned}  
        \]
        \item []综上,即证
        \[
            n \equiv 1 \pmod{9} .
        \]
    \end{enumerate}
\end{proof}

\subsection{}   %36
\begin{proof}
    \begin{enumerate}
        \item []$\forall\ p,\ \sigma(p) = p+1,\ \phi(p) = p-1,\ d(p) = 2$,有
        \[
            \sigma(p) = \phi (p) + d(p)
            \quad \Rightarrow \quad
            \sum\limits_{p\leq x} \sigma(p)
            =\sum\limits_{p\leq x} \phi(p)+
            \sum\limits_{p\leq x} d(p).
        \]
    \end{enumerate}
\end{proof}

\subsection{}   %37
\[
    \begin{aligned}
        2^1 \equiv & 2 \pmod {15}\\
        2^2 \equiv & 4 \pmod {15}\\
        2^3 \equiv & 8 \pmod {15}\\
        2^4 \equiv & 1 \pmod {15}\\
        \mbox{因此}&,2\mbox{模}15\mbox{的阶为}4.\\
    \end{aligned}
    \qquad\quad
    \begin{aligned}
        7^1 \equiv & 7  \pmod{15}\\
        7^2 \equiv & 4  \pmod{15}\\
        7^3 \equiv & 13 \pmod{15}\\
        7^4 \equiv & 1  \pmod{15}\\
        \mbox{因此}&,7\mbox{模}15\mbox{的阶为}4.\\
    \end{aligned}
    \qquad\quad
    \begin{aligned}
        8^1 \equiv & 8 \pmod{15}\\
        8^2 \equiv & 4 \pmod{15}\\
        8^3 \equiv & 2 \pmod{15}\\
        8^4 \equiv & 1 \pmod{15}\\
        \mbox{因此}&,8\mbox{模}15\mbox{的阶为}4.\\
    \end{aligned}
\]
\[
    \begin{aligned}
        4^1 \equiv & 4 \pmod{15}\\
        4^2 \equiv & 1 \pmod{15}\\
        \mbox{因此}&,4\mbox{模}15\mbox{的阶为}2.
    \end{aligned}
    \qquad\quad
    \begin{aligned}
        11^1 \equiv & 11 \pmod{15}\\
        11^2 \equiv & 1  \pmod{15}\\
        \mbox{因此}&,11\mbox{模}15\mbox{的阶为}2.
    \end{aligned}
    \qquad\quad
    \begin{aligned}
        14^1 \equiv & 14 \pmod{15}\\
        14^2 \equiv & 1  \pmod{15}\\
        \mbox{因此}&,14\mbox{模}15\mbox{的阶为}2.
    \end{aligned}
\]
\[
    \begin{aligned}
        13^1 \equiv & 13 \pmod{15}\\
        13^2 \equiv & 4  \pmod{15}\\
        13^3 \equiv & 7  \pmod{15}\\
        13^4 \equiv & 1  \pmod{15}\\
        \mbox{因此}&,13\mbox{模}15\mbox{的阶为}4.
    \end{aligned}
\]

\subsection{}   %38
\begin{enumerate}
    \item [(1)]$2$为$28$的原根.
    \[
        n = {ind}_{2}\  k\  \Leftrightarrow\  2^{n} \equiv k \pmod{29}
    \]
    \begin{table}[!ht]
        \centering
        \begin{tabular}{|c|c|c|c|c|c|c|c|c|c|c|c|c|c|c|}
        \hline
            k & 1 & 2 & 3 & 4 & 5 & 6 & 7 & 8 & 9 & 10 & 11 & 12 & 13 & 14  \\ \hline
            n & 0 & 1 & 5 & 2 & 22 & 6 & 12 & 3 & 10 & 23 & 25 & 7 & 18 & 13  \\ \hline
            k & 15 & 16 & 17 & 18 & 19 & 20 & 21 & 22 & 23 & 24 & 25 & 26 & 27 & 28  \\ \hline
            n & 27 & 4 & 21 & 11 & 9 & 24 & 17 & 26 & 20 & 8 & 16 & 19 & 15 & 14  \\ \hline
        \end{tabular}
    \end{table}
    
    \item [(2)]由$2$为$28$的原根可知
    \[
        {ind}_{2} 9 + {ind}_{2} x = {ind}_{2} 2 \pmod(28)    
    \]
    又由$(1)$得${ind}_{2} 9 = 10,\ {ind}_{2} 2=1$,有
    \[
        {ind}_{2} x = -9 \equiv 19 \pmod{28}    
        \ \Rightarrow\ 
        x\equiv 19 \pmod{29}.
    \]

    \item [(3)]由$2$为$28$的原根可知
    \[
        9\cdot {ind}_{2} x = {ind}_{2} 2 \pmod{28}
    \]
    又由$(1)$得${ind}_{2} 2=1$,有
    \begin{align*}
        9\cdot {ind}_{2} x = 1 \pmod{28}
        & \Rightarrow\ 
        {ind}_{2} x \equiv 25 \pmod{28}  \\
        & \Rightarrow\ 
        x \equiv 11 \pmod{29}.
    \end{align*}
\end{enumerate}

\subsection{}   %39
\begin{proof}
    \begin{enumerate}
        \item []
        \item []不对,证明如下:
        \[
            {457}^{911} \equiv 1 \pmod{10021}
            \quad \Leftrightarrow\quad
            \begin{cases}
                \ {457}^{911} & \equiv 1 \pmod{11}\\
                \ {457}^{911} & \equiv 1 \pmod{911}
            \end{cases}
        \]
        又
        \[
            \begin{cases}
                \ {457}^{10} & \equiv 1 \pmod{11}\\
                \ {457}^{910} & \equiv 1 \pmod{911}
            \end{cases}
            \quad \Rightarrow\quad
            \begin{cases}
                \ {457}^{911} & \equiv 6 \pmod{11}\\
                \ {457}^{911} & \equiv 1457 \pmod{911}
            \end{cases}
        \]
        故不对.
    \end{enumerate}
\end{proof}

\subsection{}   %40
\begin{enumerate}
    \item []$\phi(\phi(37))=12$,即$37$有$12$个原根.
    \item []又$2^{36}\equiv 1\pmod{37}$,且
    \[ 
        \begin{cases}
            2^{1}  \equiv  \pmod{37};   
            \qquad    
            2^{2}  \equiv  \pmod{37};\\
            2^{3}  \equiv  \pmod{37};
            \qquad
            2^{4}  \equiv  \pmod{37};\\
            2^{6}  \equiv  \pmod{37};
            \qquad
            2^{9}  \equiv  \pmod{37};\\
            2^{12} \equiv  \pmod{37};    
            \qquad
            2^{18} \equiv  \pmod{37};
        \end{cases}
        \ \Rightarrow\ 
        2\mbox{为} 37\mbox{最小原根}.
    \]
    由推论$2.7$,对于$a=2^{i}\ ((i,36)=1),\ a$模$37$的阶为$36$.即原根集合为
    \[
        \{
            \ x\  |\  1\leq x\leq 36,\ x\equiv 2^{i} \pmod{37} 
        \}
        \mbox{,且}
        i\in
        \{
            1,5,7,11,13,17,19,23,25,29,31,35
        \}
    \]
    经计算
    \[
        2^{1 } \equiv 2  \pmod{37},\             
        2^{5 } \equiv 32 \pmod{37},\         
        2^{7 } \equiv 17 \pmod{37},\         
        2^{11} \equiv 13 \pmod{37},   
    \]
    \[
        2^{13} \equiv 15 \pmod{37},\         
        2^{17} \equiv 18 \pmod{37},\         
        2^{19} \equiv 35 \pmod{37},\         
        2^{23} \equiv 5  \pmod{37},
    \]
    \[
        2^{25} \equiv 20 \pmod{37},\         
        2^{29} \equiv 24 \pmod{37},\         
        2^{31} \equiv 22 \pmod{37},\         
        2^{35} \equiv 19 \pmod{37},
    \]
    \item []即37的原根集合为
    \[
        \{
            2,5,13,15,17,18,19,20,22,24,32,35    
        \}
    \]
\end{enumerate}

\subsection{}   %41
\begin{proof}
    \begin{enumerate}
        \item []
        \item []设$(-a)$模$q$的阶为$d$.
        \[
            \begin{aligned}
                q\mid (a^p +1)
                & \Rightarrow\ 
                a^p \equiv -1 \pmod{q}\\
                & \Rightarrow\ 
                -a^p \equiv 1 \pmod{q}\\
                & \Rightarrow\ 
                {(-a)}^p \equiv 1 \pmod{q}
            \end{aligned}
            \quad \Rightarrow \quad
            d\mid p,\ d=1\mbox{或}p.
        \]
        \item [(1)]$d=1$
        \begin{align*}
            -a \equiv 1 \pmod{q}
            & \Rightarrow\ 
            a+1 \equiv 0 \pmod{q}\\
            & \Rightarrow\ 
            q\mid (a+1).
        \end{align*}
        \item [(2)]$d=p$
        \begin{align*}
            {(-a)}^p \equiv 1 \pmod{q}
            & \Rightarrow\ 
            p\mid \phi(q),\ p\mid (q-1)\\
            & \Rightarrow\ 
            2p | (q-1)\\
            & \Rightarrow\ 
            \exists\ k\in \mathbb{Z},\ 2kp = q - 1,\ q| (2kp+1)
        \end{align*}
        即证
        \[
            q\mid (a+1)
            \ \mbox{或}\ 
            q\mid (2kp + 1)
            \qquad
            (k\mbox{为某个整数}).
        \]
    \end{enumerate}
\end{proof}

\subsection{}   %42    
\begin{proof}
    \begin{enumerate}
        \item []
        \item []$6$的正因子为$1,2,3,6$,则$(a+1)$模$p$的阶为$6$等价于
        \[
            \begin{cases}
                {(a+1)}     \not \equiv 1 \pmod{p}\qquad                 
                {(a+1)}^{2} \not \equiv 1 \pmod{p}\\             
                {(a+1)}^{3} \not \equiv 1 \pmod{p}\qquad             
                {(a+1)}^{6} \equiv 1 \pmod{p}\\                 
            \end{cases}
        \]
        \item [(1)]
        \[
            (a,p)=1 \quad \Rightarrow \quad 
            (a+1) \not\equiv 1 \pmod{p}  
            \quad \Rightarrow
            p \nmid (a-1).
        \]
        \item [(2)]
        由$a^3 \equiv 1\pmod{p} , a\not\equiv 1\pmod{p} , a^2 \not\equiv 1 \pmod{p}$,得
        \[
            a^3 -1 =(a-1)(a^2+a+1) \equiv 0 \pmod{p}
            \ \Rightarrow\ 
            p\mid (a^2+a+1)
        \]
        \[
            {(a+1)}^2 = a^2 + 2a + 1 \equiv a \not\equiv 1 \pmod{p}. 
        \]
        \item [(3)]
        \[
            {(a+1)}^3 \equiv a(a+1) \equiv -a^3 \equiv -1 \pmod{p}.    
        \]
        \item [(4)]
        \[
            {(a+1)}^6 \equiv 1 \pmod{p}.    
        \]
        即证$(a+1)$模$p$的阶为$6$.
    \end{enumerate}
\end{proof}

\end{document}