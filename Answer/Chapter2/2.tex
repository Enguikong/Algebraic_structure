\documentclass[UTF8]{ctexart}
\usepackage{verbatim,amsthm,amsfonts,mathdots}
\usepackage{xeCJK,geometry,float,graphicx}
\usepackage{amsmath,amssymb,zhnumber,booktabs,setspace,tasks}
\usepackage{cases}
\usepackage{cite}
\usepackage{fancyhdr}
\usepackage{multirow}
\geometry{a4paper}
\pagestyle{fancy}
\fancyhf{}

\pagenumbering{arabic}

\begin{document}

\fancyhead[L]{En土土}
\fancyhead[C]{代数结构答案}
\fancyhead[R]{妮可}
\fancyfoot[C]{\thepage}

\section{数论初步}
\subsection{}   %1
\begin{proof}
    \begin{enumerate}
        \item []
        \item [(1)]
        \[
            \forall\ x|a,\ x|b
            \begin{cases}
                x>0 &\xrightarrow{a>0,x|a}\ x\leq a\\
                x<0 &\xrightarrow{a>0}\ x<a
            \end{cases}    
            \ \Rightarrow\ 
            x<a\ \xrightarrow{a|a,a|b}
            (a,b)=a.
        \]
        \item [(2)]
        \[
            \begin{cases}
                &(a,b)|(a,b),\ (a,b)|b\\
                \\
                &\forall\ x|(a,b),\ x|b,\ \mbox{有}x\leq (a,b).\quad (\mbox{证明同}(1))
            \end{cases}
            \ \Rightarrow\ 
            \left((a,b),b\right)=(a,b).
        \]
    \end{enumerate}
\end{proof}

\subsection{}   %2
\begin{proof}
    \begin{enumerate}
        \item []
        \item [(1)]不妨假设$\exists\ n>0,\ (n,n+1)=d>1$
        \begin{align*}
            (n,n+1)=d\ 
            \Rightarrow\ &
            \exists\ x,y\in \mathbb{Z},\ n=xd,n+1=yd\\
            \Rightarrow\ &
            1=(n+1)-n=(y-x)d>0\\
            \Rightarrow\ &
            y>x,\ (y-x)d\geq d>1\\
            \Rightarrow\ &
            \mbox{矛盾,假设不成立.}
        \end{align*}
        
        \item [(2)]可取$(n,k)$,证明如下
        \[
            \mbox{由推论2.3,取}x=1,\ a=n,\ b=k
            \mbox{,有}(n,k)=(n,n+k).    
        \]
        
    \end{enumerate}
\end{proof}

\subsection{}   %3
\begin{enumerate}
    \item [(1)]$(314,159)=1$,有解。由辗转相除法
    \begin{align*}
        314 & = 159\cdot 1 + 155\\
        159 & = 155\cdot 1 + 4\\
        155 & = 4\cdot 38 + 3\\
        4 & = 3\cdot 1 + 1  
    \end{align*}
    即
    \begin{align*}
        1 
        & = 4 - 3\cdot 1\\
        & = 4 - (155-4\cdot 38)\cdot 1\\
        & = (159-155\cdot 1)\cdot 39 - 155\\
        & = 159\cdot 39 - 155\cdot 40\\
        & = 159\cdot 39 - (314-159\cdot 1)\cdot 40\\
        & = 159\cdot 79 - 314\cdot 40.
    \end{align*}
    即$x=-40,y=79$.
    \item [(2)]$(3141,1592)=1$,有解。由辗转相除法
    \begin{align*}
        3141 & = 1592\cdot 1 + 1549\\
        1592 & = 1549\cdot 1 + 43\\
        1549 & = 43\cdot 36 + 1
    \end{align*}
    即
    \begin{align*}
        1 
        & = 1549 - 43\cdot 36\\
        & = 1549 - (1592-1549\cdot 1)\cdot 36\\
        & = 1549\cdot 37 - 1592\cdot 36\\
        & = (3141-1592\cdot 1)\cdot 37 - 1592\cdot 36\\
        & = 3141\cdot 37 - 1592\cdot 73.
    \end{align*}
    即$x=37,y=-73$.
\end{enumerate}

\subsection{}   %4
\begin{proof}
    \begin{enumerate}
        \item []
        \item [(0)]$n=1,n^3-n=0$,有$0=6\cdot 0,6|(n^3-n)$.
        \item [(1)]$n=2,n^3-n=0$,有$6=6\cdot 1,6|(n^3-n)$.
        \item [(2)]假设$n=k,k\in \mathbb{N}$时,有$6|(k^3-k)$,则$n=k+1$时有
        \begin{align*}
            {(k+1)}^3 - (k+1)\ 
            = & k^3 + 3k^2 + 2k\\
            = & (k^3-k) + 3k(k+1)
        \end{align*}
        显然有$6|(k^3-k)$,下证$6|3k(k+1)$
        \begin{enumerate}
            \item [$1^\circ$]$k=1,3k(k+1)=6$,有$6=6\cdot 1,6|3k(k+1)$
            \item [$2^\circ$]若$6|3k(k+1)$,则
            \[
                3(k+1)(k+2)=3k(k+1)+6(k+1)  
                \ \Rightarrow\ 
                6|3(k+1)(k+2)
            \]
        \end{enumerate}
        即证
        \[
            \forall\ k\in \mathbb{N},6|3k(k+1)  
            \ \Rightarrow\ 
            6|{(k+1)}^3 - (k+1)
        \]
        综上,即证
        \[
            \forall\ n>0,\ 6|(n^3-n).
        \]
    \end{enumerate}
\end{proof}

\subsection{}   %5
\begin{proof}
    \[
        \begin{cases}
            \ 3^4 \equiv 1(\bmod 10) & \Rightarrow\ 3^{4n} \equiv 1(\bmod 10)\\
            \\
            \ 10 |(3^m+1)   & \Rightarrow\ 3^m \equiv (-1) (\bmod 10)\\
        \end{cases}
        \Rightarrow
        3^{m+4n}\equiv (-1)(\bmod 10).
    \]
    即证 
    \[
        10|(3^{m+4n}+1)
    \]
\end{proof}

\subsection{}   %6
\begin{enumerate}
    \item [(1)]
    \[
        2345 = 5 \cdot  7 \cdot  67
    \]  
    \item [(2)]
    \[
        3456 = 2 \cdot  2 \cdot  2 \cdot  2 \cdot  2 \cdot  2 \cdot  2 \cdot  3 \cdot  3 \cdot  3
    \]
\end{enumerate}

\subsection{}   %7
\begin{proof}
    不妨假设$\exists\ n>0$,使得$n(n+1)=d^2$为平方数,则有
    \[
        n^2 < n(n+1) = d^2 < {(n+1)}^2
        \ \Rightarrow\ 
        n<d<n+1
    \]
    不存在相邻整数间的整数,$d$不存在,假设不成立,即证.
\end{proof}

\subsection{}   %8
\begin{proof}
    \begin{enumerate}
        \item []$n=5!+1=2\cdot 3 \cdot 4 \cdot 5 + 1$.
        \item [(1)]
        \[
            n+1 
            = 2 \cdot 3 \cdot 4 \cdot 5 + 2
            = 2 \cdot (3 \cdot 4 \cdot 5 + 1)
        \]
        \item [(2)]
        \[
            n+2 
            = 2 \cdot 3 \cdot 4 \cdot 5 + 3
            = 3 \cdot (2 \cdot 4 \cdot 5 + 1)
        \]
        \item [(3)]
        \[
            n+3 
            = 2 \cdot 3 \cdot 4 \cdot 5 + 4
            = 4 \cdot (2 \cdot 3 \cdot 5 + 1)
        \]
        \item [(4)]
        \[
            n+4 
            = 2 \cdot 3 \cdot 4 \cdot 5 + 5
            = 5 \cdot (2 \cdot 3 \cdot 4 + 1)
        \]
    \end{enumerate}
\end{proof}

\subsection{}   %9
\begin{enumerate}
    \item [(1)]$(1,1)=1|2$,方程有解,$x_0=0,y_0=2$为一组特解,故通解为
    \[
        \begin{cases}
            \ x & =\  t \\
            \ y & =\ 2 -t
        \end{cases},
        \ (t\in \mathbb{Z})
    \]
    \item [(2)]
    \item [(3)]
\end{enumerate}

\subsection{}   %10


\subsection{}   %11


\subsection{}   %12


\subsection{}   %13


\subsection{}   %14


\subsection{}   %15


\subsection{}   %16


\subsection{}   %17


\subsection{}   %18


\subsection{}   %19


\subsection{}   %20


\subsection{}   %21


\subsection{}   %22


\subsection{}   %23


\subsection{}   %24


\subsection{}   %25


\subsection{}   %26


\subsection{}   %27


\subsection{}   %28


\subsection{}   %29


\subsection{}   %30


\subsection{}   %31


\subsection{}   %32


\subsection{}   %33


\subsection{}   %34


\subsection{}   %35


\subsection{}   %36


\subsection{}   %37


\subsection{}   %38


\subsection{}   %39


\subsection{}   %40


\subsection{}   %41


\subsection{}   %42    


\end{document}