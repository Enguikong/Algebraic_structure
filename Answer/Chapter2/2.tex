\documentclass[UTF8]{ctexart}
\usepackage{verbatim,amsthm,amsfonts,mathdots}
\usepackage{xeCJK,geometry,float,graphicx}
\usepackage{amsmath,amssymb,zhnumber,booktabs,setspace,tasks}
\usepackage{cases}
\usepackage{cite}
\usepackage{fancyhdr}
\usepackage{multirow}
\geometry{a4paper}
\pagestyle{fancy}
\fancyhf{}

\pagenumbering{arabic}

\begin{document}

\fancyhead[L]{En土土}
\fancyhead[C]{代数结构答案}
\fancyhead[R]{妮可}
\fancyfoot[C]{\thepage}

\section{数论初步}
\subsection{}   %1
\begin{proof}
    \begin{enumerate}
        \item []
        \item [(1)]
        \[
            \forall\ x|a,\ x|b
            \begin{cases}
                x>0 &\xrightarrow{a>0,x|a}\ x\leq a\\
                x<0 &\xrightarrow{a>0}\ x<a
            \end{cases}    
            \ \Rightarrow\ 
            x<a\ \xrightarrow{a|a,a|b}
            (a,b)=a.
        \]
        \item [(2)]
        \[
            \begin{cases}
                &(a,b)|(a,b),\ (a,b)|b\\
                \\
                &\forall\ x|(a,b),\ x|b,\ \mbox{有}x\leq (a,b).\quad (\mbox{证明同}(1))
            \end{cases}
            \ \Rightarrow\ 
            \left((a,b),b\right)=(a,b).
        \]
    \end{enumerate}
\end{proof}

\subsection{}   %2
\begin{proof}
    \begin{enumerate}
        \item []
        \item [(1)]不妨假设$\exists\ n>0,\ (n,n+1)=d>1$
        \begin{align*}
            (n,n+1)=d\ 
            \Rightarrow\ &
            \exists\ x,y\in \mathbb{Z},\ n=xd,n+1=yd\\
            \Rightarrow\ &
            1=(n+1)-n=(y-x)d>0\\
            \Rightarrow\ &
            y>x,\ (y-x)d\geq d>1\\
            \Rightarrow\ &
            \mbox{矛盾,假设不成立.}
        \end{align*}
        
        \item [(2)]可取$(n,k)$,证明如下
        \[
            \mbox{由推论2.3,取}x=1,\ a=n,\ b=k
            \mbox{,有}(n,k)=(n,n+k).    
        \]
        
    \end{enumerate}
\end{proof}

\subsection{}   %3
\begin{enumerate}
    \item [(1)]$(314,159)=1$,有解。由辗转相除法
    \begin{align*}
        314 & = 159\cdot 1 + 155\\
        159 & = 155\cdot 1 + 4\\
        155 & = 4\cdot 38 + 3\\
        4 & = 3\cdot 1 + 1  
    \end{align*}
    即
    \begin{align*}
        1 
        & = 4 - 3\cdot 1\\
        & = 4 - (155-4\cdot 38)\cdot 1\\
        & = (159-155\cdot 1)\cdot 39 - 155\\
        & = 159\cdot 39 - 155\cdot 40\\
        & = 159\cdot 39 - (314-159\cdot 1)\cdot 40\\
        & = 159\cdot 79 - 314\cdot 40.
    \end{align*}
    即$x=-40,y=79$.
    \item [(2)]$(3141,1592)=1$,有解。由辗转相除法
    \begin{align*}
        3141 & = 1592\cdot 1 + 1549\\
        1592 & = 1549\cdot 1 + 43\\
        1549 & = 43\cdot 36 + 1
    \end{align*}
    即
    \begin{align*}
        1 
        & = 1549 - 43\cdot 36\\
        & = 1549 - (1592-1549\cdot 1)\cdot 36\\
        & = 1549\cdot 37 - 1592\cdot 36\\
        & = (3141-1592\cdot 1)\cdot 37 - 1592\cdot 36\\
        & = 3141\cdot 37 - 1592\cdot 73.
    \end{align*}
    即$x=37,y=-73$.
\end{enumerate}

\subsection{}   %4
\begin{proof}
    \begin{enumerate}
        \item []
        \item [(0)]$n=1,n^3-n=0$,有$0=6\cdot 0,6|(n^3-n)$.
        \item [(1)]$n=2,n^3-n=0$,有$6=6\cdot 1,6|(n^3-n)$.
        \item [(2)]假设$n=k,k\in \mathbb{N}$时,有$6|(k^3-k)$,则$n=k+1$时有
        \begin{align*}
            {(k+1)}^3 - (k+1)\ 
            = & k^3 + 3k^2 + 2k\\
            = & (k^3-k) + 3k(k+1)
        \end{align*}
        显然有$6|(k^3-k)$,下证$6|3k(k+1)$
        \begin{enumerate}
            \item [$1^\circ$]$k=1,3k(k+1)=6$,有$6=6\cdot 1,6|3k(k+1)$
            \item [$2^\circ$]若$6|3k(k+1)$,则
            \[
                3(k+1)(k+2)=3k(k+1)+6(k+1)  
                \ \Rightarrow\ 
                6|3(k+1)(k+2)
            \]
        \end{enumerate}
        即证
        \[
            \forall\ k\in \mathbb{N},6|3k(k+1)  
            \ \Rightarrow\ 
            6|{(k+1)}^3 - (k+1)
        \]
        综上,即证
        \[
            \forall\ n>0,\ 6|(n^3-n).
        \]
    \end{enumerate}
\end{proof}

\subsection{}   %5
\begin{proof}
    \[
        \begin{cases}
            \ 3^4 \equiv 1(\bmod 10) & \Rightarrow\ 3^{4n} \equiv 1(\bmod 10)\\
            \\
            \ 10 |(3^m+1)   & \Rightarrow\ 3^m \equiv (-1) (\bmod 10)\\
        \end{cases}
        \Rightarrow
        3^{m+4n}\equiv (-1)(\bmod 10).
    \]
    即证 
    \[
        10|(3^{m+4n}+1)
    \]
\end{proof}

\subsection{}   %6
\begin{enumerate}
    \item [(1)]
    \[
        2345 = 5 \cdot  7 \cdot  67
    \]  
    \item [(2)]
    \[
        3456 = 2 \cdot  2 \cdot  2 \cdot  2 \cdot  2 \cdot  2 \cdot  2 \cdot  3 \cdot  3 \cdot  3
    \]
\end{enumerate}

\subsection{}   %7
\begin{proof}
    不妨假设$\exists\ n>0$,使得$n(n+1)=d^2$为平方数,则有
    \[
        n^2 < n(n+1) = d^2 < {(n+1)}^2
        \ \Rightarrow\ 
        n<d<n+1
    \]
    不存在相邻整数间的整数,$d$不存在,假设不成立,即证.
\end{proof}

\subsection{}   %8
\begin{proof}
    \begin{enumerate}
        \item []$n=5!+1=2\cdot 3 \cdot 4 \cdot 5 + 1$.
        \item [(1)]
        \[
            n+1 
            = 2 \cdot 3 \cdot 4 \cdot 5 + 2
            = 2 \cdot (3 \cdot 4 \cdot 5 + 1)
        \]
        \item [(2)]
        \[
            n+2 
            = 2 \cdot 3 \cdot 4 \cdot 5 + 3
            = 3 \cdot (2 \cdot 4 \cdot 5 + 1)
        \]
        \item [(3)]
        \[
            n+3 
            = 2 \cdot 3 \cdot 4 \cdot 5 + 4
            = 4 \cdot (2 \cdot 3 \cdot 5 + 1)
        \]
        \item [(4)]
        \[
            n+4 
            = 2 \cdot 3 \cdot 4 \cdot 5 + 5
            = 5 \cdot (2 \cdot 3 \cdot 4 + 1)
        \]
    \end{enumerate}
\end{proof}

\subsection{}   %9
\begin{enumerate}
    \item [(1)]$(1,1)=1|2$,方程有解,$x_0=0,y_0=2$为一组特解,故通解为
    \[
        \begin{cases}
            \ x & =\  t \quad (t\in \mathbb{Z})\\
            \ y & =\ 2 -t
        \end{cases}
    \]
    \item [(2)]$(2,1)=1|2$,方程有解,$x_0=0,y_0=2$为一组特解,故通解为
    \[
        \begin{cases}
            \ x & =\  t \quad (t\in \mathbb{Z})\\
            \ y & =\ 2 - 2t
        \end{cases}
    \]
    \item [(3)]$(15,16)=1|17$,方程有解,$x_0=-17,y_0=17$为一组特解,故通解为
    \[
        \begin{cases}
            \ x & =\  16t - 17 \quad (t\in \mathbb{Z})\\
            \ y & =\  17 - 15t
        \end{cases}
    \]
\end{enumerate}

\subsection{}   %10
\begin{enumerate}
    \item [(1)]$(6,-15)=3|51$,方程有解,$x_0=11,y_0=1$为一组特解,故通解为
    \[
        \begin{cases}
            \ x & =\  11 - 5t \quad (t\in \mathbb{Z})\\
            \ y & =\  1 - 2t
        \end{cases}
    \]
    又要求负整数解,故$x,y<0,\ t\geq 3$,即所以负整数解为
    \[
        \begin{cases}
            \ x & =\  11 - 5t \quad (t\in \mathbb{Z},\ t\geq 3)\\
            \ y & =\  1 - 2t
        \end{cases}
    \]
    \item [(2)]$(6,15)=3|51$,方程有解,$x_0=6,y_0=1$为一组特解,故通解为
    \[
        \begin{cases}
            \ x & =\  6 + 5t \quad (t\in \mathbb{Z})\\
            \ y & =\  1 - 2t
        \end{cases}
    \]
    又要求负整数解,故$x,y<0,\ t$无解,即无负整数解.
\end{enumerate}

\subsection{}   %11
\begin{enumerate}
    \item []设需要$x$张$5$分,$y$张$1$角,$z=(30-x-y)$张$2$角五分.有
    \begin{align*}
        0.05 x + 0.1 y + 0.25 (30 - x - y) = 5\ 
        &\Leftrightarrow\ 
        x + 2 y + 5 (30 - x - y) = 100 \\
        &\Leftrightarrow\ 
        4x + 3y = 50
    \end{align*}
    \item []$(4,3)=1|50$,方程有解,$x_0=2,y_0=14$为一组特解,故通解为
    \[
        \begin{cases}
            \ x & =\  2 + 3t \quad (t\in \mathbb{Z})\\
            \ y & =\  14 - 4t
        \end{cases}
    \]
    又$x,y,z\in \mathbb{N}$,即
    \[
        \begin{cases}
            \ 2 + 3t  & \geq 0\\
            \ 14 - 4t & \geq 0\\
            \ 14 + t  & \geq 0
        \end{cases}
        \ \xrightarrow{t\in \mathbb{Z}}\ 
        t=0,1,2,3.
    \]
    即有4种方案,记$x$张$5$分,$y$张$1$角,$z$张$2$角五分,则方案为
    \[
        \begin{cases}
            \ x & = 2\\
            \ y & = 14\\
            \ z & = 14
        \end{cases}
        \qquad
        \begin{cases}
            \ x & = 5\\
            \ y & = 10\\
            \ z & = 15
        \end{cases}
        \qquad
        \begin{cases}
            \ x & = 8\\
            \ y & = 6\\
            \ z & = 16
        \end{cases}
        \qquad
        \begin{cases}
            \ x & = 11\\
            \ y & = 2\\
            \ z & = 17 
        \end{cases}
    \]
\end{enumerate}

\subsection{}   %12
\begin{enumerate}
    \item []设买了$x$个苹果,$12-x$个橘子,每个苹果$y$分钱,每个橘子$y-3$分钱,则有
    \[
        \begin{cases}
            \ 0 \geq 12-x  < x \\
            \ xy + (12-x)(y-3)  = 99
        \end{cases}
        \ \Leftrightarrow\
        \begin{cases}
            \ 6 < x \leq 12 \\
            \ x + 4y = 45
        \end{cases} 
    \]
    $(1,4)=1|45$,方程有解,$x_0=9,y_0=9$为一组特解,故通解为
    \[
        \begin{cases}
            \ x & =\  9 + 4t \quad (t\in \mathbb{Z})\\
            \ y & =\  9 - t
        \end{cases}
    \]
    又$6 < x \leq 12 $,即$t=0,\ x=9,12-x=3$,买了9个苹果和3个橘子.
\end{enumerate}

\subsection{}   %13
\[
    6k+5 \equiv 6k+1 (\bmod 4) 
\]
又$6k \equiv 6 (\bmod 4)$,有
\[
    \begin{aligned}
        6k+5 
        &\equiv 7 (\bmod 4)\\
        &\equiv 3 (\bmod 4)
    \end{aligned}
\]

\subsection{}   %14
\begin{proof}
    \begin{enumerate}
        \item []
        \item [(1)]分情况讨论$6k,6k+2,6k+3,6k+4\ (k\geq 1)$即可,不再赘述.
        \item [(2)]记素数为$p,p>3$.
        \begin{enumerate}
            \item [(a)]$p<6$,则$p=5$,成立.
            \item [(b)]$p>6$,有$(6,p)=1$,故$p$属于6的缩系,故$p$模$6$或与1或5同余.
        \end{enumerate}
    \end{enumerate}
\end{proof}

\subsection{}   %15
\begin{proof}
    \begin{enumerate}
        \item []
        \item []不妨设这两个连续的立方数为$k^3$与${(k+1)}^3$.
        \[
            \begin{aligned}
                {(k+1)}^3 - k^3 
                \equiv & 3k^2 + 3k + 1 (\bmod 3) \\
                \equiv & 1 (\bmod 3)
            \end{aligned}
        \]
    \end{enumerate}
\end{proof}

\subsection{}   %16
\begin{proof}
    \begin{enumerate}
        \item []
        \item []设该数为$A=\overline{a_n a_{n-1} \ldots a_1 a_0}$,则
        \[
            A=\sum\limits_{i=0}^{i=n} a_i\cdot 10^{i},\quad
            \sum\limits_{i=0}^{i=n} a_i \equiv 0 (\bmod 3)
        \]
        又$\forall\ k\in \mathbb{N},\ 10^{i} \equiv 1 (\bmod 3)$,故
        \begin{align*}
            A 
            \equiv & \sum\limits_{i=0}^{i=n} a_i\cdot 10^{i} (\bmod 3)\\
            \equiv & \sum\limits_{i=0}^{i=n} a_i(\bmod 3)\\
            \equiv & 0(\bmod 3)
        \end{align*}
    \end{enumerate}
\end{proof}

\subsection{}   %17
\begin{proof}
    \begin{enumerate}
        \item []
        \item [(1)]
        \[
            10 \equiv -1 (\bmod 11)
            \ \Rightarrow\ 
            10^k \equiv {(-1)}^{k} (\bmod 11)
        \]
        \item [(2)]设数为$A=\overline{a_n a_{n-1} \ldots a_1 a_0}$,则
        \[
            A \equiv 0 (\bmod 11)
            \ \Leftrightarrow\ 
            \sum\limits_{i=0}^{n} {(-1)}^{i} \cdot a_i 
            \equiv 0 (\bmod 11)
        \]
        即偶数位之和与奇数位之和的差能被$11$整除等价于该数也能被$11$整除.
    \end{enumerate}
\end{proof}

\subsection{}   %18
\begin{enumerate}
    \item [(1)]
    \[
        \begin{aligned}
            2x
            \equiv & 1 (\bmod 17)\\
            \equiv & 18 (\bmod 17)
        \end{aligned}      
        \ \xrightarrow{(2,17)=1}\ 
        x\equiv 9 (\bmod 17)
    \]
    \item [(2)]$(3,18)=3|6$,故有3组解
    由$x\equiv 2 (\bmod 6)$得原方程解为
    \[
        x\equiv 2 + 6t (\bmod 18)
        \quad (0\leq t \leq 2).
    \]
    即
    \[
        x\equiv 2,8,14 (\bmod 18)  .
    \]
    \item [(3)]$(4,18)=2|6$,故有2组解
    解$2x\equiv 3 (\bmod 9)$
    \[
        \begin{aligned}
            2x
            \equiv & 3 (\bmod 9)\\
            \equiv & 12 (\bmod 9)
        \end{aligned}
        \ \xrightarrow{(2,9)=1}\ 
        x\equiv 6(\bmod 9).
    \]
    即原方程解为
    \[
        x\equiv 6+9t (\bmod 18)\quad (t=0,1)
        \ \Rightarrow\ 
        x\equiv 6,15 (\bmod 18)  .
    \]
    
    \item [(4)]
    \[
        \begin{aligned}
            3x 
            \equiv & 1 (\bmod 17)\\
            \equiv & 18 (\bmod 17)
        \end{aligned}   
        \ \xrightarrow{(3,17)=1}\ 
        x\equiv 6(\bmod 17) .
    \]
    
\end{enumerate}

\subsection{}   %19
\begin{enumerate}
    \item [(1)]
    
    \item [(2)]
    
    \item [(3)]
    
    \item [(4)]
    
\end{enumerate}

\subsection{}   %20


\subsection{}   %21


\subsection{}   %22


\subsection{}   %23


\subsection{}   %24


\subsection{}   %25


\subsection{}   %26


\subsection{}   %27


\subsection{}   %28


\subsection{}   %29


\subsection{}   %30


\subsection{}   %31


\subsection{}   %32


\subsection{}   %33


\subsection{}   %34


\subsection{}   %35


\subsection{}   %36


\subsection{}   %37


\subsection{}   %38


\subsection{}   %39


\subsection{}   %40


\subsection{}   %41


\subsection{}   %42    


\end{document}